\documentclass[a4paper,11pt,ja=standard,xelatex]{bxjsreport}

\usepackage{geometry}
\geometry{top=25mm,bottom=25mm,left=25mm,right=25mm}
\usepackage{fontspec}
\usepackage{xeCJK}
\setmainfont{Times New Roman}
\setsansfont{Helvetica}
\setCJKmainfont[
  Path=fonts/ipaex/ipaex/,
  UprightFont=ipaexm.ttf,
  BoldFont=ipamp.ttf,
  Extension=.ttf
]{ipaexm.ttf}
\setCJKsansfont[
  Path=fonts/ipaex/ipaex/,
  UprightFont=ipaexg.ttf,
  BoldFont=ipaexg.ttf,
  Extension=.ttf
]{ipaexg.ttf}
\setCJKmonofont[
  Path=fonts/ipaex/ipaex/,
  UprightFont=ipag.ttf,
  Extension=.ttf
]{ipag.ttf}

\usepackage{graphicx}
\usepackage{amsmath}
\usepackage{amssymb}
\usepackage{amsfonts}
\usepackage{amsthm}
\usepackage{bm}
\usepackage[subrefformat=parens]{subcaption}
\usepackage{mathrsfs}
\usepackage{mathtools}
\usepackage{float}
\usepackage{comment}
\usepackage{url}
\usepackage{multirow}
\usepackage{algorithm}
\usepackage{algpseudocode}
\theoremstyle{definition}
\newtheorem{theorem}{定理}
\renewcommand{\proofname}{\textbf{証明}}

\setcounter{tocdepth}{3}
\setcounter{page}{-1}
\setlength{\parskip}{0em}
\setlength{\topsep}{0em}

\title{意思決定重視学習を用いた制約付き平均--分散ポートフォリオ最適化}
\author{野坂 健成}
\date{\today}

\begin{document}

\maketitle
\thispagestyle{empty}
\newpage

\pagenumbering{roman}
\tableofcontents
\listoffigures
\newpage
\setcounter{page}{1}
\pagenumbering{arabic}

\chapter{序論}
\section{研究背景}
\par
金融市場における資産運用では,複数の投資対象に対して資金をどのように配分するかを決定するポートフォリオ最適化問題が中心的な役割を果たしてきた.とりわけ,リスクとリターンのトレードオフを明示的に扱う平均--分散モデル(Mean--Variance Optimization; MVO)は,理論的な明快さと実務的な解釈の容易さから,現在に至るまで広く用いられている.
\par
近年では,機械学習手法の発展に伴い,将来の資産リターンをデータ駆動的に予測し,その予測結果を用いてポートフォリオ最適化を行う枠組みが一般的となっている.このようなアプローチでは,まず過去データから予測モデルを学習し,次に得られた予測値を最適化問題に入力するという\textbf{二段階法(predict-then-optimize; PTO)}の構造が採られることが多い.
\par
しかしながら,ポートフォリオ最適化問題は予測誤差に対して極めて敏感であることが知られており,期待リターンのわずかな推定誤差が投資配分を大きく変化させ,結果として投資成績の悪化を招く可能性がある.特に,実務において想定される小標本・低SNR環境や,売買制約・非負制約などを含む現実的な制約付き問題では,この問題がより顕在化する.
\par
このような背景のもと,予測精度そのものではなく,最終的に得られる意思決定の質に着目した新たな学習パラダイムが注目を集めている.

\section{問題設定}
\par
従来の PTO 型の枠組みでは,期待リターンの予測モデルは主として平均二乗誤差(MSE)などの予測誤差の最小化を目的として学習される.しかし,この学習目標は,必ずしもポートフォリオ最適化問題における投資成績の向上と一致するとは限らない.
\par
ポートフォリオ最適化において重要なのは,予測値そのものの精度ではなく,予測値に基づいて導かれる投資配分が,どれだけ望ましい意思決定を実現するかである.すなわち,予測誤差が小さいにもかかわらず,投資成績が劣化するケースや,逆に予測誤差は大きいが投資判断としては優れているケースが存在し得る.
\par
このギャップは,予測問題と最適化問題を独立に扱う二段階構造そのものに起因していると考えられる.そこで本研究では,意思決定の質を直接的に最適化対象とする学習手法に着目する.
\par
近年提案されている Decision-Focused Learning(DFL)は,予測モデルの学習段階において,下流の最適化問題を明示的に考慮し,最終的な意思決定誤差を最小化することを目的とする枠組みである.ポートフォリオ最適化問題は明確な目的関数と制約構造を持つため,DFL との親和性が高いと考えられる.
\par
本研究では,DFL の枠組みを制約付き平均--分散ポートフォリオ最適化問題に適用し,従来の PTO 手法と比較して,より良い投資判断が実現されるかを検証する.特に,実務を想定した制約条件を明示的に考慮した設定のもとで,DFL の有効性と課題を明らかにすることを目的とする.

\section{関連研究}
\par
ポートフォリオ最適化に関する研究は,Markowitz による平均--分散理論に端を発し,その後,制約付き最適化やロバスト最適化,推定誤差を考慮した手法など,多岐にわたる拡張が提案されてきた.一方で,期待リターンの推定に関しては,回帰モデルや時系列モデル,近年では機械学習を用いた手法が広く用いられている.
\par
従来の多くの研究では,予測モデルの学習とポートフォリオ最適化を独立した問題として扱う PTO の枠組みが採用されてきた.しかし,このアプローチの限界を指摘する研究も増えており,予測誤差と最適化結果との非線形な関係が問題視されている.
\par
こうした課題に対して,予測と最適化を統合的に扱う手法として,Integrating Prediction in Optimization(IPO)や Decision-Focused Learning(DFL)が提案されている.特に Butler and Kwon は,平均--分散ポートフォリオ最適化問題に対して予測モデルを統合的に学習する手法を提案し,従来手法との性能差を示した.
\par
さらに近年では,DFL を二段階最適化問題として厳密に定式化し,強双対性条件や KKT 条件を用いて再定式化する研究も進められている.これらの研究は,意思決定誤差を直接最小化するという観点から理論的に魅力的である一方,非凸性や解法の安定性,計算コストといった課題も指摘されている.
\par
本研究は,これらの先行研究を踏まえ,制約付きポートフォリオ最適化という実務的な設定において,DFL の定式化と解法設計が投資判断にどのような影響を与えるかを体系的に検討する点に特徴がある.

\section{本研究の貢献(暫定)}
\par
本研究の主な貢献は以下のとおりである.
\par
第一に,制約付き平均--分散ポートフォリオ最適化問題に対して,Decision-Focused Learning(DFL)の枠組みを適用し,その数理構造を二段階最適化問題として明示的に定式化した点である.既存研究では,制約のない設定や簡略化された問題が多く扱われてきたのに対し,本研究では,非負制約や予算制約を含む実務的な設定を考慮したモデルを構築している.
\par
第二に,強双対性条件および KKT 条件を用いた再定式化により,DFL に基づくポートフォリオ最適化問題を非凸二次計画問題として整理し,異なる再定式化が持つ数値的性質を比較可能な形で提示した点である.これにより,理論的には等価である再定式化が,数値計算上は異なる挙動を示し得ることを明確にし,解法設計や初期化の重要性を示唆する.
\par
第三に,小標本・低SNR環境を想定した数値実験を通じて,意思決定誤差を直接最小化する学習が,従来の PTO 手法と比較して,投資判断の質を改善し得る可能性を検証する点である.特に,予測誤差の最小化と投資成績の最適化が必ずしも一致しない状況において,DFL の有効性を評価する.
\par
最後に,実務を想定した制約付き設定における DFL の適用可能性と限界を整理し,今後の実データ分析や拡張に向けた課題を明確化する点も,本研究の貢献の一つである.

\section{論文構成}
\par
本論文の構成は以下のとおりである.
\par
第1章では,本研究の背景,問題設定,関連研究を整理し,本研究の目的および位置づけを明確にする.
\par
第2章では,平均--分散ポートフォリオ最適化モデルをはじめとする既存手法について概説し,従来の PTO 型アプローチおよび予測統合型手法の特徴と課題を整理する.
\par
第3章では,Decision-Focused Learning の枠組みに基づくポートフォリオ最適化問題を二段階最適化問題として定式化し,強双対性条件および KKT 条件を用いた再定式化を示す.
\par
第4章では,人工データおよび実データを用いた数値実験の設定を説明し,提案手法と既存手法との比較を通じてその特性を評価する.
\par
第5章では,本研究で得られた知見を総括するとともに,今後の課題と展望について述べる.

\chapter{既存手法}
\section{ポートフォリオ最適化モデル}
\par
本研究では,リスクとリターンのトレードオフを明示的に考慮する平均--分散ポートフォリオ最適化モデル(Mean--Variance Optimization; MVO)を基本的な投資判断モデルとして採用する.
\par
資産数を $d$ とし,投資配分ベクトルを
\[
\boldsymbol{w} = (w_1, \dots, w_d)^\top \in \mathbb{R}^d
\]
とする.また,資産リターンの期待値ベクトルを $\boldsymbol{r} \in \mathbb{R}^d$,共分散行列を $\boldsymbol{V} \in \mathbb{S}_{++}^d$ とする.ここで $\mathbb{S}_{++}^d$ は正定値対称行列の集合を表す.
\par
平均--分散モデルに基づくポートフォリオ最適化問題は,目的関数 \eqref{eq:mvo_obj} を最小化し,制約 \eqref{eq:mvo_budget}--\eqref{eq:mvo_nonneg} のもとで投資配分を決定する問題として定式化される.
\begin{subequations}
\label{eq:mvo}
\begin{align}
c(\boldsymbol{w}, \boldsymbol{r})
&= -(1-\delta)\boldsymbol{r}^\top \boldsymbol{w}
+ \frac{\delta}{2}\boldsymbol{w}^\top \boldsymbol{V}\boldsymbol{w},
\label{eq:mvo_obj}\\
\boldsymbol{1}^\top \boldsymbol{w}
&= 1,
\label{eq:mvo_budget}\\
\boldsymbol{w}
&\ge \boldsymbol{0}.
\label{eq:mvo_nonneg}
\end{align}
\end{subequations}
\par
ここで $0 \le \delta \le 1$ はリターン項とリスク項の比重を制御するトレードオフ係数であり,$\delta \to 0$ のときリターン重視,$\delta \to 1$ のときリスク重視の投資配分が得られる.目的関数 $c(\boldsymbol{w}, \boldsymbol{r})$ は,期待リターンの最大化項と分散リスクの最小化項から構成されており,制約集合は予算制約および非負制約を表している.共分散行列 $\boldsymbol{V}$ が正定値である場合,本問題は凸二次計画問題となり,グローバル最適解が一意に定まる.

\subsection{実務における推定問題}
\par
実際の運用においては,期待リターン $\boldsymbol{r}$ および共分散行列 $\boldsymbol{V}$ は未知であり,過去データから推定される.特に期待リターンの推定誤差は,最適化問題 \eqref{eq:mvo} の解に大きな影響を与えることが知られている.
\par
このため,実務では回帰モデルや時系列モデル,近年では機械学習手法を用いてリターンを予測し,その予測値を最適化問題に入力するという枠組みが一般的に用いられている.本稿では,期待リターンの推定値(予測値)を $\hat{\boldsymbol{r}}$ のようにハットを付して表す.

\section{予測統合型の手法}

\subsection{PTO アプローチ}
\par
従来の多くの研究および実務では,予測と最適化を独立した問題として扱う PTO 型のアプローチが採用されてきた.具体的には,特徴量ベクトル $\boldsymbol{x}_i \in \mathbb{R}^d$ に基づいて,資産リターンを次の線形モデルで予測する.
\begin{equation}
\hat{\boldsymbol{r}}_i(\boldsymbol{\theta}, \boldsymbol{x}_i)
= \mathrm{diag}(\boldsymbol{x}_i)\boldsymbol{\theta},
\label{eq:prediction_model}
\end{equation}
\par
ここで $\boldsymbol{\theta} \in \mathbb{R}^d$ は回帰係数である.$\boldsymbol{\theta}$ は,次の最小二乗問題を解くことで推定される.
\begin{equation}
\min_{\boldsymbol{\theta}}
\frac{1}{T} \sum_{i=1}^{T}
\left\|
\boldsymbol{r}_i - \hat{\boldsymbol{r}}_i(\boldsymbol{\theta}, \boldsymbol{x}_i)
\right\|_2^2.
\label{eq:ols}
\end{equation}
\par
得られた予測値 $\hat{\boldsymbol{r}}_i$ を用いて,ポートフォリオ最適化問題 \eqref{eq:mvo} を解くことで投資配分が決定される.

\subsection{PTO の課題}
\par
PTO アプローチでは,予測モデルの学習目的が \eqref{eq:ols} に示すように予測誤差の最小化である.一方で,最適化問題 \eqref{eq:mvo} における最終的な評価基準は投資成績,すなわち意思決定の質である.
\par
このため,予測誤差の最小化が必ずしも最適な投資判断につながるとは限らない.特に,ポートフォリオ最適化問題では,期待リターンのわずかな推定誤差が投資配分を大きく変化させることがあり,結果として投資成績が不安定になる可能性がある.この問題は,予測と最適化を分離して扱う二段階構造そのものに起因すると考えられる.

\subsection{予測統合型アプローチ(IPO)}
\par
この課題に対して,Butler and Kwon は,予測モデルの学習段階に最適化問題を統合する手法として,Integrating Prediction in Mean--Variance Portfolio Optimization(IPO)を提案した.IPO では,予測モデルのパラメータ $\boldsymbol{\theta}$ を,予測誤差ではなく,予測値に基づいて得られるポートフォリオの目的関数値を通じて更新する.これにより,予測モデルがポートフォリオ最適化問題の構造を直接考慮することが可能となる.
\par
一方で,IPO は非凸最適化問題として定式化されるため,勾配計算の計算量や局所解への収束,数値的安定性といった課題も指摘されている.

\section{本章のまとめ}
\par
本章では,平均--分散ポートフォリオ最適化モデルを基礎として,従来の PTO アプローチおよび予測統合型手法について整理した.次章では,これらの課題を踏まえ,意思決定の質を直接最適化対象とする Decision-Focused Learning に基づくポートフォリオ最適化モデルを提案する.

\chapter{提案手法}
\section{二段階最適化モデル}
\par
本章では,Decision-Focused Learning(DFL)の枠組みに基づき,制約付き平均--分散ポートフォリオ最適化問題を二段階最適化問題として定式化する.本研究では特に,リターン項とリスク項の相対的重要度を明示的に制御するため,トレードオフ係数 $\delta\in[0,1]$ を用いた修正平均--分散モデルを採用する.

\subsection{問題設定と予測モデル}
\par
時点 $i=1,\dots,T$ において,特徴量ベクトル $\boldsymbol{x}_i \in \mathbb{R}^d$ および実現リターン $\boldsymbol{r}_i \in \mathbb{R}^d$ が観測されるとする.期待リターンは,以下の線形単回帰モデルにより予測されるものとする.
\begin{equation}
\hat{\boldsymbol{r}}_i(\boldsymbol{\theta}, \boldsymbol{x}_i)
= \mathrm{diag}(\boldsymbol{x}_i)\boldsymbol{\theta},
\label{eq:dfl_prediction_new}
\end{equation}
\par
ここで $\boldsymbol{\theta} \in \mathbb{R}^d$ は学習対象となる回帰係数である.

\subsection{下位問題:制約付きポートフォリオ最適化}
\par
予測モデル \eqref{eq:dfl_prediction_new} に基づき,各時点 $i$ における投資配分(予測に基づく配分)を $\hat{\boldsymbol{w}}_i$ とし,次の制約付き最適化問題の解として定義する.
\begin{equation}
\hat{\boldsymbol{w}}_i(\boldsymbol{\theta}, \boldsymbol{x}_i)
\in
\arg\min_{\boldsymbol{w}_i \in \mathcal{S}}
c\!\left(\boldsymbol{w}_i, \hat{\boldsymbol{r}}_i(\boldsymbol{\theta}, \boldsymbol{x}_i)\right),
\label{eq:lower_level_new}
\end{equation}
\par
ただし,目的関数は次の修正平均--分散コスト関数で与えられる.
\begin{equation}
c(\boldsymbol{w}_i, \boldsymbol{r})
= -(1-\delta)\boldsymbol{r}^\top \boldsymbol{w}_i
+ \frac{\delta}{2}\boldsymbol{w}_i^\top \boldsymbol{V}_i \boldsymbol{w}_i,
\quad 0 \le \delta \le 1.
\label{eq:modified_mvo}
\end{equation}
\par
制約集合 $\mathcal{S}$ は
\begin{equation}
\mathcal{S}
= \left\{
\boldsymbol{w}_i \in \mathbb{R}^d
\ \middle|\ 
\boldsymbol{1}^\top \boldsymbol{w}_i = 1,\ 
\boldsymbol{w}_i \ge \boldsymbol{0}
\right\}
\label{eq:feasible_set_new}
\end{equation}
とする.
\par
ここで $\boldsymbol{V}_i \in \mathbb{S}_{++}^d$ は共分散行列であり,$\delta$ はリターン項とリスク項の比重を制御するパラメータである.$\delta \to 0$ のときリターン重視,$\delta \to 1$ のときリスク重視の投資配分が得られる.

\subsection{理想的な投資配分}
\par
意思決定誤差を評価する基準として,各時点 $i$ において実現リターン $\boldsymbol{r}_i$ が既知であると仮定した場合の理想的な投資配分を次のように定義する.
\begin{equation}
\boldsymbol{w}_i^*
\in
\arg\min_{\boldsymbol{w}_i \in \mathcal{S}}
c(\boldsymbol{w}_i, \boldsymbol{r}_i).
\label{eq:ideal_solution_new}
\end{equation}
\par
この投資配分は実運用では利用できないが,DFL における意思決定誤差の評価基準として用いる.

\subsection{上位問題:意思決定誤差最小化}
\par
Decision-Focused Learning では,予測モデルのパラメータ $\boldsymbol{\theta}$ を,予測誤差ではなく意思決定の質を通じて学習する.本研究では,各時点 $i$ における意思決定誤差を次のように定義する.
\begin{equation}
\ell_i(\boldsymbol{\theta})
= c\!\left(\hat{\boldsymbol{w}}_i(\boldsymbol{\theta}, \boldsymbol{x}_i), \boldsymbol{r}_i\right)
- c(\boldsymbol{w}_i^*, \boldsymbol{r}_i).
\label{eq:decision_loss_new}
\end{equation}
\par
このとき,上位問題は次の二段階最適化問題として定式化される.
\begin{subequations}
\label{eq:upper_level_new}
\begin{align}
\min_{\boldsymbol{\theta}} \quad
& \frac{1}{T}\sum_{i=1}^{T} \ell_i(\boldsymbol{\theta}),
\label{eq:upper_level_obj}\\
\text{s.t.} \quad
& \hat{\boldsymbol{w}}_i(\boldsymbol{\theta}, \boldsymbol{x}_i)
\ \text{は}\ \eqref{eq:lower_level_new}\ \text{の最適解}.
\label{eq:upper_level_constraint}
\end{align}
\end{subequations}
\par
すなわち,本研究で扱う問題は,制約付き平均--分散ポートフォリオ最適化を下位問題に含む二段階最適化問題である.

\subsection{問題の性質と再定式化への動機}
\par
定式化 \eqref{eq:upper_level_new} は,下位問題に $\arg\min$ 演算子を含むため,直接的な数値計算が困難である.また,下位問題の解を通じて定義される目的関数は一般に非凸となる.
\par
そこで本研究では,下位問題の最適性条件を用いて $\arg\min$ 演算子を除去し,単一レベルの非凸二次計画問題として再定式化する.次節では,\textbf{強双対性条件に基づく定式化(DFL-QCQP-DUAL)および KKT 条件に基づく定式化(DFL-QCQP-KKT)}について詳述する.

\section{再定式化による単一レベル最適化問題}
\par
前節で定式化した二段階最適化問題 \eqref{eq:upper_level_new} は,下位問題に $\arg\min$ 演算子を含むため,直接的な数値計算が困難である.そこで本節では,下位問題の最適性条件を用いて $\arg\min$ 演算子を除去し,単一レベルの最適化問題として再定式化する.本研究では,先行研究に従い,(1) 強双対性条件に基づく再定式化(DFL-QCQP-DUAL),(2) KKT 条件に基づく再定式化(DFL-QCQP-KKT)の2通りの定式化を考える.

\subsection{下位問題のラグランジュ関数}
\par
各時点 $i$ における下位問題 \eqref{eq:lower_level_new} を再掲する.ここでは簡単のため $\hat{\boldsymbol{r}}_i := \hat{\boldsymbol{r}}_i(\boldsymbol{\theta}, \boldsymbol{x}_i)$ とおく.
\begin{subequations}
\label{eq:lower_level_recall}
\begin{align}
\min_{\boldsymbol{w}_i \in \mathbb{R}^d} \quad
& -(1-\delta)\hat{\boldsymbol{r}}_i^\top \boldsymbol{w}_i
+ \frac{\delta}{2}\boldsymbol{w}_i^\top \boldsymbol{V}_i \boldsymbol{w}_i,
\label{eq:lower_level_recall_obj}\\
\text{s.t.} \quad
& \boldsymbol{1}^\top \boldsymbol{w}_i = 1,
\label{eq:lower_level_recall_budget}\\
& \boldsymbol{w}_i \ge \boldsymbol{0}.
\label{eq:lower_level_recall_nonneg}
\end{align}
\end{subequations}
\par
等式制約および不等式制約に対応するラグランジュ乗数を,それぞれ $\mu_i \in \mathbb{R}$,$\boldsymbol{\lambda}_i \in \mathbb{R}^d_{\ge 0}$ とすると,ラグランジュ関数は次のように与えられる.
\begin{equation}
\begin{aligned}
\mathcal{L}_i(\boldsymbol{w}_i, \mu_i, \boldsymbol{\lambda}_i)
&=
-(1-\delta)\hat{\boldsymbol{r}}_i^\top \boldsymbol{w}_i
+ \frac{\delta}{2}\boldsymbol{w}_i^\top \boldsymbol{V}_i \boldsymbol{w}_i
+ \mu_i(\boldsymbol{1}^\top \boldsymbol{w}_i - 1)
- \boldsymbol{\lambda}_i^\top \boldsymbol{w}_i.
\end{aligned}
\label{eq:lagrangian}
\end{equation}

\subsection{強双対性条件に基づく再定式化(DFL-QCQP-DUAL)}
\par
下位問題 \eqref{eq:lower_level_recall} は凸二次計画問題であり,スレーター条件が満たされるため,強双対性が成立する.このとき,原問題と双対問題の最適値が一致することから,次の条件が成り立つ.
\begin{equation}
\begin{aligned}
-(1-\delta)\hat{\boldsymbol{r}}_i^\top \boldsymbol{w}_i
+ \frac{\delta}{2}\boldsymbol{w}_i^\top \boldsymbol{V}_i \boldsymbol{w}_i
\le
\mu_i,
\end{aligned}
\label{eq:dual_value}
\end{equation}
\par
さらに,ラグランジュ関数の一階条件より,
\begin{equation}
\delta \boldsymbol{V}_i \boldsymbol{w}_i
-(1-\delta)\hat{\boldsymbol{r}}_i
=
-\mu_i \boldsymbol{1}
+ \boldsymbol{\lambda}_i.
\label{eq:dual_stationarity}
\end{equation}
\par
以上を用いることで,二段階最適化問題 \eqref{eq:upper_level_new} は,次の単一レベル最適化問題として再定式化される.
\begin{subequations}
\label{eq:dfl_dual}
\begin{align}
\min_{\boldsymbol{\theta}, \{\boldsymbol{w}_i,\mu_i,\boldsymbol{\lambda}_i\}_{i=1}^{T}}
\quad
& \frac{1}{T}\sum_{i=1}^{T}
\left(
-(1-\delta)\boldsymbol{r}_i^\top \boldsymbol{w}_i
+ \frac{\delta}{2}\boldsymbol{w}_i^\top \boldsymbol{V}_i \boldsymbol{w}_i
\right),
\label{eq:dfl_dual_obj}\\
\text{s.t.} \quad
& \boldsymbol{1}^\top \boldsymbol{w}_i = 1,
\qquad i=1,\dots,T,
\label{eq:dfl_dual_budget}\\
& \boldsymbol{w}_i \ge \boldsymbol{0},
\qquad i=1,\dots,T,
\label{eq:dfl_dual_nonneg}\\
& \boldsymbol{\lambda}_i \ge \boldsymbol{0},
\qquad i=1,\dots,T,
\label{eq:dfl_dual_lambda_nonneg}\\
& -(1-\delta)\hat{\boldsymbol{r}}_i^\top \boldsymbol{w}_i
+ \frac{\delta}{2}\boldsymbol{w}_i^\top \boldsymbol{V}_i \boldsymbol{w}_i
\le \mu_i,
\qquad i=1,\dots,T,
\label{eq:dfl_dual_value}\\
& \delta \boldsymbol{V}_i \boldsymbol{w}_i
-(1-\delta)\hat{\boldsymbol{r}}_i
= -\mu_i \boldsymbol{1}
+ \boldsymbol{\lambda}_i,
\qquad i=1,\dots,T.
\label{eq:dfl_dual_stationarity}
\end{align}
\end{subequations}
\par
この定式化を DFL-QCQP-DUAL と呼ぶ.

\subsection{KKT 条件に基づく再定式化(DFL-QCQP-KKT)}
\par
別の再定式化として,下位問題 \eqref{eq:lower_level_recall} の KKT 条件をすべて制約として組み込む方法を考える.KKT 条件は以下から構成される.
\par
一次の最適性条件
\begin{equation}
\delta \boldsymbol{V}_i \boldsymbol{w}_i
-(1-\delta)\hat{\boldsymbol{r}}_i
= -\mu_i \boldsymbol{1}
+ \boldsymbol{\lambda}_i,
\label{eq:kkt_stationarity}
\end{equation}
\par
実行可能性条件
\begin{subequations}
\label{eq:kkt_feasible}
\begin{align}
\boldsymbol{1}^\top \boldsymbol{w}_i &= 1,
\label{eq:kkt_budget}\\
\boldsymbol{w}_i &\ge \boldsymbol{0},
\label{eq:kkt_w_nonneg}\\
\boldsymbol{\lambda}_i &\ge \boldsymbol{0}.
\label{eq:kkt_lambda_nonneg}
\end{align}
\end{subequations}
\par
相補性条件
\begin{equation}
\boldsymbol{\lambda}_i \odot \boldsymbol{w}_i = \boldsymbol{0}.
\label{eq:kkt_complementarity}
\end{equation}
\par
これらを用いることで,次の単一レベル最適化問題が得られる.
\begin{subequations}
\label{eq:dfl_kkt}
\begin{align}
\min_{\boldsymbol{\theta}, \{\boldsymbol{w}_i,\mu_i,\boldsymbol{\lambda}_i\}_{i=1}^{T}}
\quad
& \frac{1}{T}\sum_{i=1}^{T}
\left(
-(1-\delta)\boldsymbol{r}_i^\top \boldsymbol{w}_i
+ \frac{\delta}{2}\boldsymbol{w}_i^\top \boldsymbol{V}_i \boldsymbol{w}_i
\right),
\label{eq:dfl_kkt_obj}\\
\text{s.t.} \quad
& \boldsymbol{1}^\top \boldsymbol{w}_i = 1,
\qquad i=1,\dots,T,
\label{eq:dfl_kkt_budget}\\
& \boldsymbol{w}_i \ge \boldsymbol{0},
\qquad i=1,\dots,T,
\label{eq:dfl_kkt_nonneg}\\
& \boldsymbol{\lambda}_i \ge \boldsymbol{0},
\qquad i=1,\dots,T,
\label{eq:dfl_kkt_lambda_nonneg}\\
& \delta \boldsymbol{V}_i \boldsymbol{w}_i
-(1-\delta)\hat{\boldsymbol{r}}_i
= -\mu_i \boldsymbol{1}
+ \boldsymbol{\lambda}_i,
\qquad i=1,\dots,T,
\label{eq:dfl_kkt_stationarity}\\
& \boldsymbol{\lambda}_i \odot \boldsymbol{w}_i = \boldsymbol{0},
\qquad i=1,\dots,T.
\label{eq:dfl_kkt_complementarity}
\end{align}
\end{subequations}
\par
この定式化を DFL-QCQP-KKT と呼ぶ.

\subsection{二つの再定式化の性質}
\par
DFL-QCQP-DUAL \eqref{eq:dfl_dual} と DFL-QCQP-KKT \eqref{eq:dfl_kkt} は,理論的には下位問題の最適性条件を表現しており,同一の解集合を持つ.一方で,数値計算の観点からは両者は異なる特徴を持つ.DFL-QCQP-DUAL は相補性条件を含まない一方で,非線形な不等式制約を含む.DFL-QCQP-KKT は相補性条件という非線形制約を含むが,制約の構造はより直接的である.
\par
これらの違いにより,実際の数値計算では,初期化やソルバーの探索挙動に依存して,解の安定性や収束性に差が生じる可能性がある.これらの点については,第4章の数値実験において詳しく検証する.

\section{計算量および解法に関する注意}
\par
本章で示した DFL-QCQP-DUAL および DFL-QCQP-KKT は,いずれも単一レベルの最適化問題として定式化されているが,その計算量的性質には注意が必要である.
\par
まず,両定式化はいずれも非線形制約を含む\textbf{非凸二次計画問題(non-convex QCQP)}であり,一般にはグローバル最適解を保証する多項式時間アルゴリズムは存在しない.そのため,数値計算においては,局所最適解への収束や初期化への依存といった問題が生じ得る.
\par
次に,DFL-QCQP-DUAL と DFL-QCQP-KKT は理論的には下位問題の最適性条件を表現しており,同一の解集合を持つ.しかしながら,数値計算の観点からは両者は異なる特徴を持つ.DFL-QCQP-DUAL は相補性条件を含まない一方で,非線形な不等式制約を含む.一方,DFL-QCQP-KKT は相補性条件という非線形制約を含むが,制約構造はより直接的である.この違いにより,ソルバーの探索挙動や収束性に差が生じる可能性がある.
\par
さらに,本研究で扱う問題は,上位問題と下位問題が強く結合しているため,初期解の選択が数値計算の安定性に影響を与えることがある.特に,異なる初期化を用いた場合に,異なる局所解に収束する可能性がある点には留意が必要である.
\par
以上を踏まえ,本研究では,特定の解法に依存した結論を導くのではなく,複数の再定式化および初期化条件のもとで数値実験を行い,提案手法の挙動と特性を総合的に評価する立場を取る.

\chapter{数値実験}
\par
本章では,第3章で提案した Decision-Focused Learning(DFL)に基づくポートフォリオ最適化手法について,実データを用いた数値実験によりその特性を検証する.本節では,実験の基本方針,使用データ,共分散推定方法,比較手法,ならびに初期化およびソルバー設定について述べる.

\section{実データ実験の設定}

\subsection{実験の基本方針}
\par
第3章で示した DFL-QCQP-DUAL および DFL-QCQP-KKT は,いずれも非凸な二次計画問題として定式化される.このため,数値計算においては初期解やソルバーの探索挙動に依存して,異なる局所解に収束する可能性がある.
\par
本研究では,特定の解法や初期化に依存した性能主張を行うことを目的とせず,意思決定重視学習としての挙動および特性を,実務を想定した設定のもとで評価することを目的とする.そのため,以下の方針に基づいて実験を設計する.
\begin{itemize}
  \item 実データおよび実務的な制約条件を用いる
  \item 比較手法間で,予測モデル,制約条件,評価指標を統一する
  \item 初期化およびソルバー設定を明示し,比較の公平性と再現性を確保する
\end{itemize}

\subsection{使用データおよび学習・再バランス設定}
\par
実データ実験では,短期タクティカル・アセットアロケーション(Tactical Asset Allocation; TAA)を想定し,週次リターンデータを用いる.分析期間は 2006 年 1 月から 2025 年 12 月とし,以下の 4 資産を投資対象とする.
\begin{itemize}
  \item SPY:米国株式(S\&P 500)
  \item GLD:金
  \item EEM:新興国株式
  \item TLT:米国長期国債
\end{itemize}
\par
各時点 $t$ における特徴量 $\boldsymbol{x}_t$ として,直近 26 週のリターン平均を用いる.この設定は,短期的な市場トレンドを捉えつつ,過度なノイズへの反応を抑制することを意図したものであり,短期 TAA において一般的に用いられる時間スケールに基づいている.
\par
モデルの学習および更新は,以下のローリング手順により行う.
\begin{itemize}
  \item 直近 26 週のデータを用いて予測モデルのパラメータを推定
  \item 推定したパラメータを次の 4 週間にわたって固定して使用
  \item 以降,同様の手順を繰り返す
\end{itemize}
\par
この再バランス頻度は,推定誤差の増幅や過度な売買を避けつつ,市場環境の変化に一定程度追随することを目的として設定している.なお,本章の実験では,学習窓長や再バランス頻度の最適化は行わず,すべての比較手法に対して同一の設定を用いることで,モデル構造および学習方式の違いに焦点を当てる.これらのパラメータに対する感度分析については,後続の補足実験において検討する.

\subsection{共分散行列の推定(OAS $\times$ EWMA)}
\par
本研究では,ポートフォリオ最適化に用いる共分散行列 $\boldsymbol{V}_t$ を,時間減衰を考慮した標本共分散行列に対して Oracle Approximating Shrinkage(OAS)を適用する方法により推定する.
\par
まず,時点 $t$ における EWMA 共分散行列 $\boldsymbol{S}_t$ を次式で定義する.
\begin{equation}
\boldsymbol{S}_t
= (1-\alpha)\sum_{k=0}^{L-1}
\alpha^k
(\boldsymbol{r}_{t-k} - \bar{\boldsymbol{r}}_t)
(\boldsymbol{r}_{t-k} - \bar{\boldsymbol{r}}_t)^\top,
\label{eq:ewma_cov}
\end{equation}
\par
ここで,$L=13$ はローリング窓長,$\alpha=0.97$ は時間減衰率であり,実務における業界慣例に基づき固定する.また,$\bar{\boldsymbol{r}}_t$ は同窓内の平均リターンを表す.
\par
次に,OAS に基づく縮小共分散行列 $\boldsymbol{V}_t$ を次のように定義する.
\begin{equation}
\boldsymbol{V}_t
= (1-\phi_t)\boldsymbol{S}_t
+ \phi_t
\frac{\mathrm{tr}(\boldsymbol{S}_t)}{d}
\boldsymbol{I},
\label{eq:oas_cov}
\end{equation}
\par
ここで $\phi_t\in[0,1]$ は縮小強度であり,OAS により解析的に決定される.具体的には,$d$ 次元,共分散推定に用いる(実効的な)サンプルサイズを $n_{\mathrm{eff}}$ とすると,
\begin{equation}
\phi_t
= \min\left\{
1,\ 
\frac{\left(1-\frac{2}{d}\right)\mathrm{tr}(\boldsymbol{S}_t^2)+\mathrm{tr}(\boldsymbol{S}_t)^2}
{\left(n_{\mathrm{eff}}+1-\frac{2}{d}\right)\left(\mathrm{tr}(\boldsymbol{S}_t^2)-\frac{\mathrm{tr}(\boldsymbol{S}_t)^2}{d}\right)}
\right\}.
\label{eq:oas_phi}
\end{equation}
\par
EWMA のように観測に重みを付ける場合,$n_{\mathrm{eff}}$ は「時間減衰率 $\alpha$ に対応する実効サンプルサイズ」とみなせる.本研究のように有限窓 $L$ を用いるとき,重みを正規化した $\tilde{w}_k := \frac{(1-\alpha)\alpha^k}{1-\alpha^L}$($k=0,\dots,L-1$)に対して
\begin{equation}
n_{\mathrm{eff}}
:= \frac{1}{\sum_{k=0}^{L-1}\tilde{w}_k^2}
= \frac{(1-\alpha^L)^2}{(1-\alpha)^2}\cdot\frac{1-\alpha^2}{1-\alpha^{2L}}
= \frac{1+\alpha}{1-\alpha}\cdot\frac{(1-\alpha^L)^2}{1-\alpha^{2L}}.
\label{eq:neff_ewma}
\end{equation}
\par
特に $L$ が十分大きい場合には $n_{\mathrm{eff}} \approx \frac{1+\alpha}{1-\alpha}$ となる.OAS は,小標本下において標本共分散行列の推定誤差を抑制しつつ,分散構造の情報を保持する点で知られており,本研究のような短期ローリング設定と高い親和性を持つ.

\subsection{トレードオフ係数 $\delta$ の設定}
\par
本研究では,第3章で導入した修正平均--分散コスト関数
\begin{equation}
c(\boldsymbol{w}, \boldsymbol{r})
= -(1-\delta)\boldsymbol{r}^\top \boldsymbol{w}
+ \frac{\delta}{2}\boldsymbol{w}^\top \boldsymbol{V}\boldsymbol{w}
\label{eq:delta_cost}
\end{equation}
を用いる.
\par
実データ実験においては,リスクとリターンのバランスを取った中庸的な設定として $\delta=0.5$ をデフォルト値として固定する.$\delta$ を学習対象や探索変数とせず固定することで,(i)解法の不安定性とパラメータ感度を分離する,(ii)定式化および解法(dual / KKT)の影響に焦点を当てる,という実験設計上の意図を明確にする.

\subsection{比較手法}
\par
実データ実験では,予測モデル,制約条件,評価指標を可能な限り統一したうえで,以下の手法を比較対象として用いる.
\par
\textbf{二段階法(PTO)の基準:}
\begin{itemize}
  \item OLS + MVO(PTO):期待リターンを最小二乗誤差で推定し,得られた予測値 $\hat{\boldsymbol{r}}$ を平均--分散最適化に入力して配分を決定する,実務でも最も標準的な二段階法.
\end{itemize}
\par
\textbf{予測統合型(IPO)/end-to-end 系:}
\begin{itemize}
  \item IPO(analytic):ポートフォリオ最適化を予測パラメータ学習に統合した IPO を,制約を外した設定で解析的に解ける形として実装したもの(本研究では初期化・アンカーとしても利用する).
  \item IPO-GRAD:下位最適化問題を通じて勾配を伝播させる勾配ベースの統合学習(end-to-end)手法.
\end{itemize}
\par
\textbf{Decision-Focused Learning 系:}
\begin{itemize}
  \item SPO+:意思決定誤差の近似勾配を用いる代表的な DFL 手法(ベースライン).
  \item DFL-QCQP-DUAL / DFL-QCQP-KKT:第3章で導出した dual / KKT 再定式化に基づく提案手法.
\end{itemize}
\par
\textbf{ベンチマーク(運用戦略):}
\begin{itemize}
  \item Buy-and-Hold(SPY):株式市場に対する単純な長期保有戦略.
  \item 等分散投資($1/N$):推定誤差に依存しない頑健な基準配分.
  \item 時系列モメンタム(TSMOM-SPY):TAA 的な比較として代表的なトレンドフォロー戦略.
\end{itemize}

\subsection{初期化およびソルバー設定}
\par
第3章で述べたとおり,本研究で扱う最適化問題は非凸であり,初期化およびソルバー設定が数値結果に影響を与える可能性がある.本研究では,非線形最適化ソルバーとして KNITRO を用い,以下の方針で設定を行う.
\begin{itemize}
  \item 内点法(Interior/Direct)を採用する
  \item BFGS による近似ヘッセ行列を使用する
  \item 反復回数および収束許容誤差を十分厳しく設定する
  \item スレッド数を 1 に固定し,再現性を確保する
\end{itemize}
\par
また,DFL 系手法では,現時点の実データ実験では初期値として $\boldsymbol{\theta}=\boldsymbol{0}$ を用い(必要に応じて投資配分は等配分,補助変数は $0$ から開始する),初期化に起因する差をできるだけ抑える.一方で,後続の実験では IPO の解析解(IPO-analytic)に基づく初期化やアンカーも併せて検討し,初期化の影響も含めて評価する.

\section{実験結果}

\subsection{実データによるベースライン比較}
\par
本節では,第3章で導出した提案手法である Decision-Focused Learning(DFL)に基づく
DFL-QCQP モデル(dual 定式化および KKT 定式化)について,
\textbf{初期解(warm-start)および正則化項(アンカー・罰則項)を一切導入しないベースライン設定}
の下で,既存手法との比較評価を行う.
\par
比較対象として,従来の予測誤差最小化に基づく OLS(PTO),
予測統合型の手法である IPO(analytic / gradient),
ならびに代表的な Decision-Focused Learning 手法である SPO+ を含める.
また,単純な投資戦略として $1/N$ ポートフォリオ,
時系列モメンタム(TSMOM; SPY),Buy-and-Hold(SPY)をベンチマークとして用いる.
\par
評価指標として,年率リターン(ann\_return),最終資産(terminal\_wealth),
リスク調整後指標(Sharpe 比・Sortino 比),
リスク指標(年率ボラティリティ,最大ドローダウン,CVaR(95\%)),
および運用上重要な売買回転率(平均ターンオーバー)を用いる.
なお,本実験では取引コスト(bps)を加味した実行ベースの損益系列に基づき指標を算出している
(詳細は第4.1節を参照).

%========================
% Figure: cum return
%========================
\par
まず,図\ref{fig:baseline_cumreturn}に,2006年から2025年までの全期間における各手法の累積リターン推移を示す.
図\ref{fig:baseline_cumreturn}より,提案手法(DFL-QCQP)は dual / KKT のいずれの定式化においても,
多くの比較手法を上回る累積リターンを示していることが確認できる.
特に,長期にわたる運用期間を通じて,リターンの上振れだけでなく,
下落局面における毀損の抑制も一定程度観察される.

\begin{figure}[H]
\centering
\includegraphics[width=0.95\linewidth]{figs/baseline_cum_return.png}
\caption{累積リターン推移(2006--2025,ベースライン設定)}
\label{fig:baseline_cumreturn}
\end{figure}

%========================
% Table 1: summary metrics
%========================
\par
次に,表\ref{tab:baseline_summary}に主要なパフォーマンス指標をまとめる.
提案手法は年率リターン・最終資産において上位であるだけでなく,
Sharpe 比・CVaR(95\%) といったリスク調整後指標においても良好な値を示している.
このことは,単なるリターンの増大ではなく,
\textbf{リスク制御を伴った意思決定の改善}が実現されている可能性を示唆する.
\par
また,ターンオーバーの観点では,
提案手法(DFL-QCQP)は IPO-analytic や OLS と同程度の水準に収まっており,
性能改善が売買過多のみによって生じているとは言い難い.
さらに,dual と KKT の比較では,
KKT 定式化が多くの指標で僅かに優位である一方,
両者の性能差は大きくなく,定式化の違いが投資性能に与える影響は限定的であることも示唆される.

\begin{table}[H]
\centering
\caption{パフォーマンス比較(2006--2025,ベースライン設定)}
\label{tab:baseline_summary}
\begin{tabular}{lrrrrrrrr}
\hline
Model & Ann.\ Return & Terminal & Sharpe & Sortino & Ann.\ Vol & MaxDD & CVaR95 & Turnover \\
\hline
DFL-QCQP-kkt  & \textbf{11.86} & \textbf{8.36} & \textbf{0.78} & \textbf{0.73} & 15.22 & -35.13 & 5.01 & 23.98 \\
DFL-QCQP-dual & 11.54 & 7.79 & 0.75 & 0.72 & 15.45 & -36.39 & 5.03 & 23.96 \\
IPO-analytic  & 10.46 & 5.90 & 0.60 & 0.57 & 17.38 & -30.43 & 5.68 & 24.24 \\
Buy\&Hold(SPY)& 10.40 & 5.67 & 0.57 & 0.52 & 18.16 & -58.36 & 6.23 & \textbf{0.00} \\
SPO+          &  8.10 & 3.62 & 0.45 & 0.44 & 17.91 & -33.37 & 5.93 & 20.69 \\
1/N           &  7.16 & 3.65 & 0.63 & 0.59 & \textbf{11.43} & -33.49 & \textbf{3.67} & \textbf{0.00} \\
TSMOM(SPY)    &  7.25 & 3.66 & 0.61 & 0.46 & 11.92 & \textbf{-24.66} & 4.26 & 1.45 \\
IPO-GRAD      &  7.48 & 3.13 & 0.40 & 0.37 & 18.49 & -55.15 & 6.44 & 20.19 \\
OLS           &  6.29 & 2.42 & 0.33 & 0.30 & 19.06 & -55.68 & 6.73 & 20.79 \\
\hline
\end{tabular}
\end{table}

%========================
% Table 2: significance test
%========================
\par
最後に,表\ref{tab:baseline_significance}に,
Sharpe 比,CVaR(95\%),最終資産(terminal wealth)に関して,
ブートストラップに基づく統計的検定(対応あり)を行った結果を示す.
表中の ``1'' は,提案手法が比較対象に対して \textbf{5\%水準で有意に優れている}ことを表し,
``0'' は有意差が確認できないことを表す.
\par
表\ref{tab:baseline_significance}より,
提案手法は CVaR(95\%) および最終資産において複数の比較手法に対して一貫して有意な改善を示している.
一方で,Sharpe 比については必ずしも全ての比較で有意差が成立しておらず,
提案手法の改善が「平均リターンの上昇」だけではなく,
\textbf{下方リスク(tail risk)の抑制を通じた改善}として現れている可能性が高い.
なお,TSMOM(SPY) に対しては有意差が確認できない指標も残っており,
トレンドフォロー型の単純戦略が本実験設定において一定の競争力を持つことも示唆される.

\begin{table}[H]
\centering
\caption{統計的有意性(5\%水準,1: DFLが有意に優位,0: 有意差なし)}
\label{tab:baseline_significance}
\begin{tabular}{lccc}
\hline
Compared model (vs DFL-QCQP-kkt) & Sharpe & CVaR95 & Final wealth \\
\hline
1/N            & 0 & 1 & 1 \\
Buy\&Hold(SPY) & 0 & 1 & 1 \\
IPO-GRAD       & 1 & 1 & 1 \\
IPO-analytic   & 0 & 1 & 1 \\
OLS            & 1 & 1 & 1 \\
SPO+           & 0 & 1 & 1 \\
TSMOM(SPY)     & 0 & 0 & 0 \\
\hline
\end{tabular}
\end{table}

\par
以上より,特別な初期化や正則化を行わないベースライン設定においても,
提案手法(DFL-QCQP)は既存の予測主導型手法(OLS)および代表的 DFL 手法(SPO+),
ならびに IPO 系手法に対して競争力のある投資性能を示すことが確認できる.
次節では,提案手法における dual / KKT の定式化差が数値計算(収束性・安定性)および
投資性能に与える影響について,より詳細に検討する.

\subsection{Dual formulation と KKT formulation の数値的比較}
\par
本節では,提案手法において用いた非凸 QCQP に対する二つの定式化,dual formulation と KKT formulation の数値的性質を比較する.
両者は理論的には等価であり,同一の最適解集合を持つことが示されるが,非凸最適化問題を数値的に解く際には,定式化の違いがソルバの収束挙動,計算コスト,および探索経路への依存性に影響を与える可能性がある.
\par
本節の目的は投資パフォーマンスの優劣を比較することではなく,
\textbf{どちらの定式化が数値計算としてより安定的かつ実務的に採用可能か}
を明らかにすることである.
\par
そのため,以下の 4 つの観点から比較を行う.
\begin{enumerate}
  \item 数値計算の信頼性(solver status)
  \item 計算コスト(特に worst-case を表す p90)
  \item 解の同等性(理論的等価性の数値的検証)
  \item 探索経路摂動に対する頑健性(非凸性への耐性)
\end{enumerate}

\subsubsection{数値計算の信頼性}
\par
まず,各定式化に対する数値計算の信頼性を評価する.ここでは,全期間を通じた rolling retraining において,各リバランス時点で解かれた最適化問題に対し,ソルバが返した終了ステータスを集計した.
\par
評価区分は以下の通りである.
\begin{itemize}
  \item OK:最適性条件を満たして終了
  \item Warning:実行可能解は得られたが,収束判定に関する警告あり(例:xtol に基づく相対改善停止,near-optimal 判定)
  \item No-solution:実行可能解が得られなかったケース
\end{itemize}
\par
なお,Warning に分類されたケースについても,全てのケースで実行可能解が返却されており,制約違反量はいずれも数値誤差レベル($10^{-8}$ 以下)に留まっている.

\begin{table}[H]
\centering
\caption{Solver status distribution}
\label{tab:formulation_status}
\begin{tabular}{lrrr}
\hline
Formulation & OK (\%) & Warning (\%) & No-solution (\%) \\
\hline
Dual & 95.0 & 5.0 & 0.0 \\
KKT  & 88.1 & 11.9 & 0.0 \\
\hline
\end{tabular}
\end{table}

\par
表\ref{tab:formulation_status}より,両定式化ともに No-solution は一度も発生しておらず,数値計算としての基本的な安定性は共通して確保されていることが分かる.
一方で,KKT formulation では Warning の割合がやや高いが,その内訳を確認すると,多くは最適性改善が所定の閾値以下となった段階での終了であり,実行可能性や制約充足の破綻を示すものではない.

\begin{figure}[H]
\centering
\includegraphics[width=0.95\linewidth]{figs/formulation_warning_breakdown_dual_vs_kkt.png}
\caption{warning の内訳(dual vs KKT)}
\label{fig:formulation_status_counts}
\end{figure}

\subsubsection{計算コストの比較}
\par
次に,計算コストの観点から両定式化を比較する.非凸最適化では平均計算時間よりも稀に発生する難ケースでの計算時間が実務上重要であるため,本研究では 90 パーセンタイル(p90)を主要指標として用いる.
\par
評価指標は以下の通りである.
\begin{itemize}
  \item Median:代表的な計算時間
  \item Mean:参考値
  \item p90:最悪 10\% の計算時間を表す指標
\end{itemize}

\begin{table}[H]
\centering
\caption{Computation time statistics (seconds)}
\label{tab:formulation_elapsed}
\begin{tabular}{lrrr}
\hline
Formulation & Median & Mean & p90 \\
\hline
Dual & 1.20 & 1.65 & 2.99 \\
KKT  & 0.75 & 0.95 & 1.71 \\
\hline
\end{tabular}
\end{table}

\par
さらに,計算時間分布全体を可視化するため,箱ひげ図を図\ref{fig:formulation_elapsed}に示す.
\begin{figure}[H]
\centering
\includegraphics[width=0.92\linewidth]{figs/formulation_elapsed_ok_vs_warning.png}
\caption{計算時間の分布(dual vs KKT;OK と Warning を分けて表示)}
\label{fig:formulation_elapsed}
\end{figure}
\par
中央値および平均値では両定式化に大きな差は見られないものの,p90 においては KKT formulation の方が小さく,計算時間分布の裾が抑制されている.
すなわち,KKT formulation は worst-case においても計算時間が過度に増大しにくい特性を持つ.

\subsubsection{解の同等性の検証}
\par
次に,dual formulation と KKT formulation が数値的にも同一の問題を解いているかを検証する.評価指標として,(i) ポートフォリオ重みの差($L_1/L_2$ 距離),(ii) 下位 MVO 問題の目的関数値差,(iii) 学習時の目的関数値差を用いる.

\begin{table}[H]
\centering
\caption{Solution agreement (Dual vs KKT)}
\label{tab:formulation_agreement}
\begin{tabular}{lrr}
\hline
Metric & Median & p90 \\
\hline
MVO cost diff & $1.2\times10^{-10}$ & $5.3\times10^{-3}$ \\
Train objective diff & $2.8\times10^{-6}$ & $1.3\times10^{-3}$ \\
Weight $L_1$ distance & $3.4\times10^{-8}$ & 1.17 \\
Weight $L_2$ distance & $2.2\times10^{-8}$ & 0.77 \\
\hline
\end{tabular}
\end{table}

\par
中央値ではすべての指標が数値誤差レベルに留まっており,dual と KKT は大多数の期間でほぼ同一の解を与えている.
一方,p90 付近では一時的に差が大きくなるケースが存在するが,これは非凸性に起因する探索経路の差異によるものであり,特定の定式化が系統的に劣ることを示すものではない.

\subsubsection{探索経路摂動に対する頑健性}
\par
非凸問題では,同一問題であっても初期化(探索経路)により到達点が変化し得る.そこで補助変数初期値を摂動し,探索経路依存性を評価した.ここでは摂動強度を simplex 上の $L_1$ 距離
\[
\Delta \triangleq \lVert \boldsymbol{w}^{(0)}_t - \boldsymbol{w}_{\mathrm{base},t} \rVert_1
\]
で実効的に較正し,\textbf{$\Delta \approx 0.2$(weak 相当)}の条件で複数 seed を用いて学習・評価を行った.なお,本結果は IPO 解析解に基づく初期化(init-ipo)を用いた設定で得られたものである.
\par
図\ref{fig:perturb_weak_boxplots}に,Sharpe / CVaR / Terminal wealth の箱ひげ図(seed 間の分布)を示す.また,図\ref{fig:perturb_weak_wealth}に累積リターンの overlay(seed 間比較)を示す.

\begin{figure}[H]
  \centering
  \begin{subfigure}[t]{0.32\linewidth}
    \centering
    \includegraphics[width=\linewidth]{figs/perturb_weak_sharpe_boxplot.png}
    \caption{Sharpe}
  \end{subfigure}
  \begin{subfigure}[t]{0.32\linewidth}
    \centering
    \includegraphics[width=\linewidth]{figs/perturb_weak_cvar_boxplot.png}
    \caption{CVaR(95\%)}
  \end{subfigure}
  \begin{subfigure}[t]{0.32\linewidth}
    \centering
    \includegraphics[width=\linewidth]{figs/perturb_weak_terminal_wealth_boxplot.png}
    \caption{Terminal wealth}
  \end{subfigure}
  \caption{探索経路摂動($\Delta \approx 0.2$)下での性能分布(dual vs kkt)}
  \label{fig:perturb_weak_boxplots}
\end{figure}

\begin{figure}[H]
  \centering
  \begin{subfigure}[t]{0.92\linewidth}
    \centering
    \includegraphics[width=\linewidth]{figs/perturb_weak_cumret_dual.png}
    \caption{dual}
  \end{subfigure}
  \vspace{0.6em}
  \begin{subfigure}[t]{0.92\linewidth}
    \centering
    \includegraphics[width=\linewidth]{figs/perturb_weak_cumret_kkt.png}
    \caption{kkt}
  \end{subfigure}
  \caption{探索経路摂動($\Delta \approx 0.2$)下での累積リターン overlay(seed 間比較)}
  \label{fig:perturb_weak_wealth}
\end{figure}

\par
観測された傾向($\Delta \approx 0.2$ 条件)を以下にまとめる.
\begin{itemize}
  \item 性能分布(summary\_table の seed 集計)では,KKT の Sharpe 平均が Dual より高く(Dual: 0.5232, KKT: 0.6737),Terminal wealth も KKT が大きい(Dual: 4.192, KKT: 6.160).
  \item 一方で Warning 発生は KKT の方が多い(平均 Warning count: Dual 8.0, KKT 27.3).ただし,その内訳は主に NEARLY\_OPT / NO\_IMPROVE 型であり,No-solution が発生しない事実と併せると,ここでの Warning は\textbf{“失敗”ではなく“収束判定の違い(停止条件の差)”}として扱うのが妥当である.
  \item cumulative wealth overlay では,Dual の方が seed によるばらつきが大きい(上振れもあるが下振れも大きい)という形が視覚的に確認できる.一方 KKT は線群が比較的まとまる.
\end{itemize}

\par
この段階で言えるのは「KKT が常に優越する」ではなく,より厳密には
\begin{itemize}
  \item KKT:計算時間分布(p90)の観点で安定(X.2),摂動下の性能分布も比較的安定(本実験)
  \item Dual:Warning は少ない(X.1)が,摂動下で分散が増えやすい可能性(本実験)
\end{itemize}
という\textbf{トレードオフ構造}である.したがって採用判断としては,\textbf{KKT formulation を優先}するのが論理的に自然である.

\subsubsection{小結}
\par
本節では,dual formulation と KKT formulation の数値的性質を多角的に比較した.その結果,
\begin{itemize}
  \item 両定式化は解の同等性を数値的にも満たしている
  \item 計算時間の p90(worst-case)の観点で,KKT formulation は安定的な傾向を示す
  \item 探索経路摂動($\Delta \approx 0.2$)下では,KKT formulation は性能分布が比較的安定であり,dual より高い中央値・平均を示す一方,Warning は増加する
\end{itemize}
\par
以上の結果を踏まえ,以降の数値実験では KKT formulation を主たる定式化として採用する.

\subsection{初期解導入の効果(DFL-CF init)}
\par
本節では,提案手法 DFL-QCQP(KKT formulation)において制約なし DFL-MVO の解析解(以下,DFL-CF 解)を初期解として与えた場合の効果を検証する.
\par
非凸最適化問題においては,探索開始点が収束先の局所解および収束速度に影響を与えることが知られている.本研究では,
\begin{itemize}
  \item 初期解を与えない場合(no init)
  \item 制約なし DFL-MVO の解析解を初期化として与える場合(DFL-CF init)
\end{itemize}
を比較し,性能・安定性・計算コストの観点から初期解導入の影響を定量的に評価する.

\subsubsection{累積リターンの比較}
\par
図\ref{fig:init_effect_cumreturn}に,初期解導入の有無および比較手法を含めた累積リターン推移を示す.
図\ref{fig:init_effect_cumreturn}より,DFL-QCQP-KKT は初期解の有無にかかわらず比較手法を上回る累積リターンを示す一方,初期解(DFL-CF 解)を導入することで,特に後半期間において累積リターンが上方にシフトしていることが確認できる.
また,初期解として用いた DFL-CF 解そのものよりも,再最適化後の DFL-QCQP の方が高い最終パフォーマンスを達成しており,良好な初期解は最終解を固定するものではなく,探索を有利な領域へ導く役割を果たしていることが示唆される.

\begin{figure}[H]
\centering
\includegraphics[width=0.95\linewidth]{figs/init_effect_wealth.png}
\caption{累積リターン推移(初期解あり vs なし,および比較手法)}
\label{fig:init_effect_cumreturn}
\end{figure}

\subsubsection{全期間の性能指標比較}
\par
次に,全期間における主要評価指標を表\ref{tab:init_effect_summary}に示す.

\begin{table}[H]
\centering
\caption{初期解導入の有無による性能比較(全期間)}
\label{tab:init_effect_summary}
\begin{tabular}{lrrrrrrr}
\hline
Model & Ann.\ Return (\%) & Terminal & Sharpe & Sortino & Ann.\ Vol (\%) & MaxDD (\%) & CVaR$_{95}$ (\%) \\
\hline
DFL-QCQP-KKT (init=DFL-CF) & 12.16 (+0.30) & 8.89 (+0.53) & 0.80 (+0.02) & 0.75 (+0.02) & 15.17 ($-$0.05) & $-$34.73 (+0.40) & $-$4.97 (+0.04) \\
DFL-QCQP-KKT (no init)     & 11.86          & 8.36          & 0.78          & 0.73          & 15.22          & $-$35.13          & $-$5.01          \\
IPO-GRAD                   & 11.51          & 7.30          & 0.67          & 0.65          & 17.26          & $-$29.75          & $-$5.51          \\
DFL-CF(解析解のみ)       & 10.46          & 5.90          & 0.60          & 0.57          & 17.38          & $-$30.43          & $-$5.68          \\
SPO+                       & 10.29          & 5.60          & 0.57          & 0.57          & 17.95          & $-$33.25          & $-$5.77          \\
OLS                        &  6.29          & 2.42          & 0.33          & 0.30          & 19.06          & $-$55.68          & $-$6.73          \\
\hline
\end{tabular}
\end{table}

\par
括弧内は初期解なし(no init)との差分(init $-$ no init)である.表\ref{tab:init_effect_summary}から,初期解導入によりリターン・リスク調整後指標が一貫して改善していることが読み取れる(年率リターン:+0.30\%,Sharpe / Sortino:ともに +0.02,CVaR$_{95}$:+0.04).年率ボラティリティはほぼ不変であり,リスク水準を上げることなくリターン効率が向上している.改善幅は過度に大きくはないが,全指標で符号が揃っている点は重要であり,初期解導入が体系的に探索結果を改善していることを示している.

\subsubsection{計算時間への影響}
\par
表\ref{tab:init_effect_time}に,計算時間の比較(平均・最大)および Warning 件数を示す.

\begin{table}[H]
\centering
\caption{計算時間の比較(初期解あり vs なし)}
\label{tab:init_effect_time}
\begin{tabular}{lrrr}
\hline
Model & Mean fit time (sec) & Max fit time (sec) & Warning count \\
\hline
DFL-QCQP-KKT (init=DFL-CF) & 0.73 & 2.66 & 33 \\
DFL-QCQP-KKT (no init)     & 0.95 & 8.89 & 31 \\
\hline
\end{tabular}
\end{table}

\par
初期解を導入することで,平均計算時間は短縮され,最大計算時間も大幅に抑制されている.これは,初期解が探索初期段階における不利な方向への移動を抑制し,比較的早期に良好な局所解近傍へ到達している可能性を示唆する.

\subsubsection{小結}
\par
本節では,制約なし DFL-MVO の解析解を初期解として導入した場合の効果を検証した.その結果,
\begin{itemize}
  \item 初期解導入により,提案手法の主要性能指標が一貫して改善
  \item 初期解そのものを上回る性能が,制約付き再最適化によって達成される
  \item 計算時間の短縮という実務的利点も同時に得られる
\end{itemize}
ことが確認された.以上より,良好な初期解の導入は,非凸 DFL 問題において性能・安定性・計算効率を同時に改善する有効な戦略であると結論づけられる.

\section{考察}
\par
(後続節)

\section{補足分析}
\par
(後続節)

\chapter{結論}
\par
本研究では,制約付き平均--分散ポートフォリオ最適化を下位問題として含む Decision-Focused Learning(DFL)を対象に,意思決定誤差に基づく学習問題を二段階最適化問題として定式化した.さらに,下位問題の最適性条件に基づき,強双対性条件に基づく再定式化(DFL-QCQP-DUAL)および KKT 条件に基づく再定式化(DFL-QCQP-KKT)を導出し,単一レベルの非凸二次計画問題(QCQP)として整理した.
\par
また,実務を想定した週次リターンの実データ設定のもとで,データ分割,共分散推定,比較手法,初期化およびソルバー設定を明示し,提案手法の挙動を評価するための実験設計を示した.今後は,第4章で得られる実験結果を踏まえ,定式化の違いが収束性・解の安定性・投資成績に与える影響を整理するとともに,より多資産・より現実的な制約を含む設定への拡張を検討する.

\chapter*{参考文献}
\addcontentsline{toc}{chapter}{\numberline{}参考文献}
\begin{thebibliography}{99}

\bibitem{lee2024returnprediction}
Lee, J.,
\newblock Return Prediction for Mean-Variance Portfolio Selection: How Decision-Focused Learning Shapes Forecasting Models,
\newblock Proceedings of \ldots\ (to appear), 2024.
\par
意思決定重視学習(Decision-Focused Learning; DFL)を平均--分散ポートフォリオ選択に適用し,予測モデルが意思決定構造にどのような影響を与えるかを分析した研究.
\par\bigskip

\bibitem{kim2025covariance}
Kim, J., Tae, I., Lee, Y.,
\newblock Estimating Covariance for Global Minimum Variance Portfolio: A Decision-Focused Learning Approach,
\newblock arXiv preprint arXiv:2508.10776, 2025.
\par
DFL の枠組みを用いてグローバル最小分散ポートフォリオ(GMVP)の共分散推定問題を再定式化し,意思決定損失最小化の観点から理論的および実証的検討を行った研究.
\par\bigskip

\par\bigskip
\noindent\textbf{一般的な Decision-Focused Learning 理論}
\par\medskip

\bibitem{butlerkwon2021ipo}
Butler, J.\ B., Kwon, S.\ J.,
\newblock Integrating Prediction in Mean-Variance Portfolio Optimization,
\newblock 2021.
\par
予測モデルとポートフォリオ最適化問題を統合的に扱う予測統合型最適化(Integrated Prediction and Optimization; IPO)の枠組みを提案した先行研究.
\par\bigskip

\bibitem{mwp2024dflsurvey}
M.\ W.\ P.\ et al.,
\newblock Decision-Focused Learning: Foundations, State of the Art, Benchmark and Future Opportunities,
\newblock arXiv preprint, 2024.
\par
Decision-Focused Learning の基礎理論から最新動向,代表的ベンチマークおよび今後の研究課題までを包括的に整理した総説論文.
\par\bigskip

\bibitem{butlerkwon_pessimistic}
Butler, J.\ B., Kwon, S.\ J.,
\newblock Decision-Focused Predictions via Pessimistic Bilevel Optimization: Complexity and Algorithms,
\newblock Journal / arXiv, Year.
\par
DFL を悲観的二段階最適化(pessimistic bilevel optimization)として定式化し,計算複雑性および再定式化手法を理論的に解析した研究.
\par\bigskip

\bibitem{shah2022locallyoptimized}
Shah, S., et al.,
\newblock Learning Locally Optimized Decision Losses,
\newblock Advances in Neural Information Processing Systems (NeurIPS), 2022.
\par
最適化問題を含む意思決定損失の微分可能近似を提案し,DFL の一般的学習枠組みを拡張した研究.
\par\bigskip

\par\bigskip
\noindent\textbf{共分散推定・Shrinkage モデル}
\par\medskip

\bibitem{ledoitwolf2004}
Ledoit, O., Wolf, M.,
\newblock A Well-Conditioned Estimator for Large-Dimensional Covariance Matrices,
\newblock Journal of Multivariate Analysis, 2004.
\par
高次元環境における共分散行列の shrinkage 推定法を提案した基礎的研究.
\par\bigskip

\bibitem{bodnar2021dynamic}
Bodnar, T.,
\newblock Dynamic Shrinkage Estimation of the High-Dimensional Minimum-Variance Portfolio,
\newblock arXiv preprint arXiv:2106.02131, 2021.
\par
GMV ポートフォリオに対する動的 shrinkage 共分散推定モデルを提案した研究.
\par\bigskip

\bibitem{bodnar2022two}
Bodnar, T., Parolya, N., Thors\'en, E.,
\newblock Two Is Better Than One: Regularized Shrinkage of Large Minimum Variance Portfolio,
\newblock arXiv preprint arXiv:2202.06666, 2022.
\par
共分散推定とポートフォリオ重み正則化を同時に行う二重 shrinkage モデルを理論的に分析した研究.
\par\bigskip

\bibitem{tan2020cv}
Tan, V., Zohren, S.,
\newblock Estimation of Large Financial Covariances: A Cross-Validation Approach,
\newblock arXiv preprint arXiv:2012.05757, 2020.
\par
交差検証を用いた大規模金融共分散推定手法を提案した研究.
\par\bigskip

\par\bigskip
\noindent\textbf{伝統的ポートフォリオ最適化理論}
\par\medskip

\bibitem{markowitz1952}
Markowitz, H.,
\newblock Portfolio Selection,
\newblock The Journal of Finance, 1952.
\par
平均--分散ポートフォリオ理論を提唱した近代ポートフォリオ理論の原典.
\par\bigskip

\bibitem{wikipedia_mpt}
Wikipedia contributors,
\newblock Modern Portfolio Theory,
\newblock \url{https://en.wikipedia.org/wiki/Modern_portfolio_theory}.
\par
近代ポートフォリオ理論の概要および基本概念をまとめた補足資料.
\par\bigskip

\par\bigskip
\noindent\textbf{関連応用研究}
\par\medskip

\bibitem{anis2025cardinality}
Anis, H.\ T., et al.,
\newblock End-to-End, Decision-Based, Cardinality-Constrained Portfolio Optimization,
\newblock Journal of \ldots, 2025.
\par
End-to-end 学習と DFL を組み合わせ,組合せ制約付きポートフォリオ最適化問題を扱った研究.
\par\bigskip

\bibitem{kim2025semidfl}
Kim, Juhyeong,
\newblock Semi-Decision-Focused Learning with Deep Ensembles: A Practical Framework for Robust Portfolio Optimization,
\newblock Proceedings of the International Conference on Learning Representations (ICLR), 2025.
\par
Deep Ensemble と DFL を統合した実務志向のロバストポートフォリオ最適化手法を提案した研究.

\end{thebibliography}

\newpage
\chapter*{謝辞}
\addcontentsline{toc}{chapter}{\numberline{}謝辞}
本研究をご指導くださった高野祐一准教授をはじめ,議論に協力してくださった研究室の皆様に深く感謝いたします.
\end{document}
