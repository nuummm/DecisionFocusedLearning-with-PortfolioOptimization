\documentclass[a4paper,10.5pt]{jreport}

%========================
% 先輩テンプレ準拠(platex + dvipdfmx)
%========================
\usepackage{test}

% 余白(geometry は使用しない)
% A4 / 上下左右 25mm 相当(TARGET の geometry 設定と同値)
\setlength{\topmargin}{-0.4mm}
\setlength{\oddsidemargin}{-0.4mm}
\setlength{\evensidemargin}{-0.4mm}
\setlength{\textwidth}{160mm}
\setlength{\textheight}{247mm}
\setlength{\headheight}{0mm}
\setlength{\headsep}{0mm}
\setlength{\footskip}{0mm}

\usepackage[dvipdfmx]{graphicx}
\usepackage[dvipdfmx,table]{xcolor}
\usepackage{amsmath}
\usepackage{amssymb}
\usepackage{amsfonts}
\usepackage{amsthm}
\usepackage{bm}
\usepackage[subrefformat=parens]{subcaption}
\usepackage{mathrsfs}
\usepackage{mathtools}
\usepackage{float}
\usepackage{comment}
\usepackage{url}
\usepackage{multirow}
\usepackage{algorithm}
\usepackage{algpseudocode}
\theoremstyle{definition}
\newtheorem{theorem}{定理}[chapter]
\newtheorem{proposition}{命題}[chapter]
\renewcommand{\proofname}{\normalfont 証明}

% 目次深さ
\setcounter{tocdepth}{3}
\newcounter{savedtocdepth}

%========================
% タイトル情報(test.sty 経由で \maketitle)
%========================
\title{意思決定重視学習を用いたポートフォリオ最適化}
\author{野坂 健成}
\studentid{学籍番号:202211661}
\advisor{指導教員:高野 祐一 准教授}
\majorfield{経営工学主専攻}
\programfield{}
\yearandmonth{令和7年1月21日提出}
\date{}

\begin{document}

%========================
% 表紙
%========================
\maketitle

%========================
% 目次・図目次(前付)
%========================
\pagenumbering{roman}
\setcounter{tocdepth}{3}
\tableofcontents
\listoffigures
\listoftables
\pagebreak
\setcounter{page}{1}
\pagenumbering{arabic}

%==========================================================================================
% 第1章:序論
%========================
\chapter{序論}
%========================
% 研究背景
%========================
\section{研究背景}
\par
金融市場における資産運用では,複数資産への資金配分を決定するポートフォリオ最適化が中心的役割を担う.
ポートフォリオ最適化では一般に,期待リターンとリスクのトレードオフを考慮した投資配分の決定が求められる.
近年は機械学習の発展により,各資産の将来リターンをデータ駆動的に推定し,
その推定値を最適化問題に入力して投資配分を得る枠組みが一般的である.
このとき,多くの場合で予測誤差を最小化する目的で学習される.
しかし,この学習目標が必ずしも投資成績(意思決定品質)の改善に直結するとは限らないことが指摘されている\cite{lahoud2025,mandi2024}.
このギャップの一因は,予測と最適化を独立に扱う二段階構造にある.
すなわち,わずかな予測誤差であっても最適化段階で増幅され,最終的な投資配分が大きく変化し,
結果としてリスク調整後成績などの評価指標が悪化する場合があり得る.
この現象は,推定不確実性が大きい局面や制約を含む現実的設定において,より顕在化しやすい\cite{chopraziemba1993}.
\par
以上を背景として,近年,予測精度そのものではなく,後続の最適化問題を通じて定義される意思決定の質を直接最適化する
意思決定重視学習(Decision-Focused Learning;  DFL)が注目されている\cite{mandi2024}.
本論文では,「推定値に基づいて得られる投資配分が,理想的配分と比べてどれだけ目的関数値を悪化させるか」を意思決定誤差(機会損失)と呼ぶ.
DFL では,予測モデルの学習目標を「推定値と実現値の誤差」ではなく,この意思決定誤差を直接最小化するように定める.
%========================
% 問題設定
%========================
\section{問題設定}
\par
時点 $i=1,\dots,T$ において,各資産に関する特徴量を $\boldsymbol{x}_i$,当該期間の実現リターンを $\boldsymbol{r}_i$ とする.
\par
予測モデルはパラメータ $\boldsymbol{\theta}$ を持ち,$\boldsymbol{x}_i$ に基づいて
期待リターンの推定値 $\hat{\boldsymbol{r}}_i(\boldsymbol{\theta},\boldsymbol{x}_i)$ を出力する.

\par
各時点における投資配分 $\hat{\boldsymbol{w}}_i(\boldsymbol{\theta})$ は,
推定値 $\hat{\boldsymbol{r}}_i$ を入力として解かれる制約付きポートフォリオ最適化の最適解として定まる.
ここで $\mathcal{S}$ はポートフォリオ制約で定まる実行可能集合,$c(\cdot,\cdot)$ は目的関数を表す(具体形は第2章で与える).
\[
\hat{\boldsymbol{w}}_i(\boldsymbol{\theta})
\in \arg\min_{\boldsymbol{w}\in\mathcal{S}}
c(\boldsymbol{w},\hat{\boldsymbol{r}}_i(\boldsymbol{\theta},\boldsymbol{x}_i)).
\]
\par
一方,比較の基準として,もし実現リターン $\boldsymbol{r}_i$ が事前に既知であったと仮定した場合に得られる理想的な投資配分を
\[
\boldsymbol{w}_i^\star \in \arg\min_{\boldsymbol{w}\in\mathcal{S}} c(\boldsymbol{w},\boldsymbol{r}_i)
\]
と定義する.

\par
DFL は,予測誤差ではなく,最終的に得られる投資配分の良さに基づく意思決定誤差を学習目標として定める.
本研究では,各時点の意思決定誤差を
\[
\ell_i(\boldsymbol{\theta})
=
c(\hat{\boldsymbol{w}}_i(\boldsymbol{\theta}),\boldsymbol{r}_i)
-
c(\boldsymbol{w}_i^\star,\boldsymbol{r}_i)
\]
と定義し,$\frac{1}{T}\sum_{i=1}^T \ell_i(\boldsymbol{\theta})$ を最小化する.
すなわち,本研究で扱う DFL は,制約付きポートフォリオ最適化を下位問題に含む二段階最適化として定式化される.
%========================
% 関連研究
%========================
\section{関連研究}
\par
ポートフォリオ最適化は Markowitz による平均--分散ポートフォリオ最適化モデル(Mean--Variance Optimization; MVO)\cite{markowitz1952} を起点として,
制約付き最適化,ロバスト最適化,推定誤差を考慮した手法など多岐に拡張されてきた.
MVO は,期待リターンとリスクを明示的に扱えるため,理論的にも実務的にも解釈しやすい基準モデルとして広く用いられてきた.
一方で,期待リターン推定は回帰・時系列モデルから機械学習まで幅広く研究されている.
平均推定誤差が最適配分に与える影響の大きさは,Chopra and Ziemba\cite{chopraziemba1993} により定量的に示されている.

\par
従来の多くの研究および実務では,予測モデルを予測誤差(MSE など)の最小化で学習する予測精度重視学習(Prediction-Focused Learning; PFL)に基づき,
得られた推定値を最適化問題に入力して投資配分を決定する.
このように,推定誤差の最小化により推定モデルを学習し,得られた推定値を最適化問題に入力して投資配分を得る二段階構造が広く用いられている\cite{lahoud2025}.
しかし,PFL による学習目的は最終的な最適化目的と必ずしも一致しないため,予測誤差と投資成績の乖離が生じ得る\cite{mandi2024}.

\par
この課題に対し,予測と最適化を統合的に扱う枠組みとして DFL が提案されている\cite{mandi2024}.
DFL では,学習過程に最適化問題を組み込み,最適化解を介して定義される意思決定誤差を直接最小化する.
さらに,平均--分散ポートフォリオ選択に DFL を適用し,推定モデルが意思決定構造によりどのように影響を受けるかを分析する研究も報告されている\cite{lee2025}.
\par
また,平均--分散ポートフォリオ最適化に対して推定と最適化を統合的に扱う予測統合型手法として IPO が提案されている\cite{butlerkwon2023}.
加えて,DFL を悲観的二段階最適化として厳密に捉える理論的整理\cite{bucarey2024} や,数値安定性の改善に関するアルゴリズム的検討\cite{shah2022} も進められている.
%========================
% 本研究の貢献
%========================
\section{本研究の貢献}
\par
本研究の主な貢献は以下のとおりである.
\begin{enumerate}
  \item 制約付き平均--分散ポートフォリオ最適化を下位問題に含む DFL を対象に,下位問題の最適性条件に基づく2通りの同値再定式化を導出し,比較可能な形で整理した.
  \item 実データを用いて,理論的に同値な再定式化であっても数値計算上の挙動(収束,安定性,初期化依存)が異なり得ることを示し,解法設計上の論点を明確化した.
  \item 週次TAAを想定した実務的設定において,PFL(OLS+MVO)と比較し,提案手法が予測精度の改善では説明できない意思決定品質(Sharpe, CVaR)を実現し得ることを実証した.
  \item 提案手法が出力した投資配分と損益系列に基づき,機会損失の分布,売買量・切替頻度,条件数レジーム別比較などの性質分析を通じて,「効く局面」および挙動の解釈可能性を整理した.
\end{enumerate}
%========================
% 論文構成
%========================
\section{論文構成}
\par
本論文の構成は以下のとおりである.
第2章では,既存手法として PFL および予測統合型手法を整理する.
第3章では,提案手法を定式化し,DFL-OPT-D / DFL-OPT-K の再定式化を示す.
第4章では,実データを用いた数値実験により提案手法の特性を検証する.
第5章では,結論と今後の課題を述べる.

%===========================================================================================
% 第2章:既存手法
%========================
\chapter{既存手法}
%========================
% ポートフォリオ最適化モデル
%========================
\section{ポートフォリオ最適化モデル}
\par
本研究では,Markowitz \cite{markowitz1952} による平均--分散ポートフォリオ最適化モデル(Mean--Variance Optimization, MVO)を基準モデルとして採用する.
本モデルは,期待リターンと分散リスクのトレードオフを明示的に表現できるため,解釈しやすい枠組みとして広く用いられてきた.
\par
資産数を $d$ とし,投資配分ベクトルを
\[
\boldsymbol{w} = (w_1, \dots, w_d)^\top \in \mathbb{R}^d
\]
とする.また,期待リターンベクトルを $\boldsymbol{r} \in \mathbb{R}^d$,共分散行列を $\boldsymbol{V} \in \mathbb{S}_{++}^d$ とする.
ここで $\mathbb{S}_{++}^d$ は正定値対称行列の集合を表す.
\par
平均--分散モデルに基づくポートフォリオ最適化問題は,目的関数 \eqref{eq:mvo_obj} を最小化し,制約 \eqref{eq:mvo_budget}--\eqref{eq:mvo_nonneg} のもとで
投資配分を決定する問題として定式化される.
\begin{align}
c(\boldsymbol{w}, \boldsymbol{r})
&:= -(1-\delta)\boldsymbol{r}^\top \boldsymbol{w}
+ \frac{\delta}{2}\boldsymbol{w}^\top \boldsymbol{V}\boldsymbol{w},
\label{eq:mvo_obj}\\
\boldsymbol{1}^\top \boldsymbol{w}
&= 1,
\label{eq:mvo_budget}\\
\boldsymbol{w}
&\ge \boldsymbol{0}.
\label{eq:mvo_nonneg}
\end{align}
\par
ここで $0\le\delta\le 1$ は,期待リターン項と分散リスク項の比重を直接制御するパラメータである.
最小化問題として扱うために期待リターン項に負号を付しており,期待リターンを大きくするほど $-\boldsymbol{r}^\top \boldsymbol{w}$ が小さくなる.
また $\frac{\delta}{2}\boldsymbol{w}^\top \boldsymbol{V}\boldsymbol{w}$ は分散リスクを表す.共分散行列 $\boldsymbol{V}$ が正定値である場合,
本問題は凸二次計画問題となり,大域的最適解が一意に定まる.なお,標準的な平均--分散モデルで用いられるリスク回避係数 $\gamma>0$ を用いた
表現 $-\boldsymbol{r}^\top\boldsymbol{w}+\frac{\gamma}{2}\boldsymbol{w}^\top\boldsymbol{V}\boldsymbol{w}$ は,
係数の単調変換($\gamma=\frac{\delta}{1-\delta}$, $\delta\in[0,1)$)により \eqref{eq:mvo_obj} と対応する.

\par
実際の運用においては,期待リターン $\boldsymbol{r}$ および共分散行列 $\boldsymbol{V}$ は未知であり,過去データから推定される.特に期待リターンの推定誤差は,
最適化問題 \eqref{eq:mvo_obj}--\eqref{eq:mvo_nonneg} の解に大きな影響を与えることが知られている\cite{chopraziemba1993}.
\par
このため,実務では回帰モデルや時系列モデル,近年では機械学習手法を用いてリターンを推定し,その推定値を最適化問題に入力するという枠組みが一般的に
用いられている\cite{lahoud2025}.本稿では,期待リターンの推定値を $\hat{\boldsymbol{r}}$ として表す.
\par
本章では,期待リターンは特徴量に対して線形に表現できると仮定し,特徴量ベクトル $\boldsymbol{x}_i \in \mathbb{R}^d$ に基づいて次期リターンの推定値を次の形で与える.
\begin{equation}
\hat{\boldsymbol{r}}_i(\boldsymbol{\theta}, \boldsymbol{x}_i)
= \mathrm{diag}(\boldsymbol{x}_i)\boldsymbol{\theta},
\label{eq:prediction_model}
\end{equation}
\par
ここで $\boldsymbol{\theta}\in\mathbb{R}^{d}$ は回帰係数である.$\boldsymbol{\theta}$ は,次の最小二乗問題を解くことで推定される.
\begin{equation}
\min_{\boldsymbol{\theta}}
\frac{1}{T} \sum_{i=1}^{T}
\left\|
\boldsymbol{r}_i - \hat{\boldsymbol{r}}_i(\boldsymbol{\theta}, \boldsymbol{x}_i)
\right\|_2^2.
\label{eq:ols}
\end{equation}
\par
得られた推定値 $\hat{\boldsymbol{r}}_i$ を用いて,ポートフォリオ最適化問題 \eqref{eq:mvo_obj}--\eqref{eq:mvo_nonneg} を解くことで投資配分が決定される.
本稿では,このように推定誤差の最小化を学習目的とし,推定と最適化を分離して扱う枠組みを PFL と呼ぶ\cite{lahoud2025,mandi2024}.
%========================
% 予測統合型アプローチ
%========================
\section{予測統合型アプローチ}
\par
\par
推定と最適化の分離に起因する課題に対し,Butler and Kwon \cite{butlerkwon2023} は,推定モデルの学習段階にポートフォリオ最適化問題を直接組み込む予測統合型アプローチとして,
Integrating Prediction in Mean--Variance Portfolio Optimization(IPO)を提案した.
\par
IPO は,推定モデルのパラメータを,予測誤差ではなく「推定に基づいて得られる最終的な意思決定の良さ」を通じて更新する枠組みである.例えば,各時点 $t$ において
推定値 $\hat{\boldsymbol{r}}_t(\boldsymbol{\theta})$ が与えられたとき,ポートフォリオ最適化の最適解を $\boldsymbol{w}^*(\hat{\boldsymbol{r}}_t)$ とすると,
IPO は次の最適化問題として表現できる.
\begin{equation}
\min_{\boldsymbol{\theta}}\;
\mathbb{E}_{t}\!\left[
c\!\left(\boldsymbol{w}^{*}(\hat{\boldsymbol{r}}_t(\boldsymbol{\theta})),\,\hat{\boldsymbol{r}}_t(\boldsymbol{\theta})\right)
\right].
\label{eq:ipo_upper}
\end{equation}
\begin{equation}
 \boldsymbol{w}^{*}(\hat{\boldsymbol{r}}_t)
\in
\arg\min_{\boldsymbol{w}\in\mathcal{S}}
c(\boldsymbol{w},\,\hat{\boldsymbol{r}}_t).
\label{eq:ipo_lower}
\end{equation}
\par
この定式化では,推定モデルのパラメータ $\boldsymbol{\theta}$ は予測誤差ではなく,推定値に基づいて得られるポートフォリオの目的関数値を通じて更新される.
すなわち,学習の評価基準は「推定がどれだけ正確か」ではなく,「推定に基づく投資配分がどれだけ良いか」によって定義される.
\par
一方で, IPO は一般に非凸最適化問題として定式化され,勾配ベースの手法により学習が行われる.そのため,計算コストの増大や,初期値に依存した局所解への収束,
制約付き設定における数値的安定性といった課題も指摘されている.特に,非負制約や予算制約を含む実務的な平均--分散最適化問題に対しては,学習の安定性が必ずしも保証されない場合がある.
%========================
% 意思決定考慮型アプローチ
%========================
\section{意思決定考慮型アプローチ}
\par
推定と最適化の分離に起因する課題に対し,Elmachtoub and Grigas \cite{elmachtoubgrigas2022} は Smart Predict--then--Optimize(SPO+)を提案した. 
SPO+ は,従来の二段階法の枠組みを維持しつつ,最適化問題の構造を反映した surrogate loss を用いて推定モデルを学習する手法である.
\par
この点で SPO+ は,純粋な prediction-focused learning よりも意思決定を意識した学習を行う一方,最適化問題を学習ループに完全に組み込む DFL とは異なり,
二段階構造に基づくアプローチと位置づけられる.
\par
本研究では,SPO+ を「二段階法における意思決定考慮型手法」の代表例として比較対象に含める.

\par
%========================
% 共分散行列の推定(補足)
%========================
\section*{共分散行列の推定}
\par
本研究では期待リターンの推定に焦点を当て,共分散行列の推定そのものは研究対象としない.共分散推定誤差がポートフォリオ最適化の安定性に与える影響は大きく, 
shrinkage 推定\cite{chen2010oas} や時間減衰を考慮した推定法\cite{jpm2006riskmetrics} など,これまでに多くの手法が提案されてきた.
\par
本研究では,これら既存研究の知見に基づき,小標本・短期ローリング設定において数値的安定性が高いことが知られている手法として, 
Oracle Approximating Shrinkage(OAS)\cite{chen2010oas} と 
Exponentially Weighted Moving Average(EWMA)\cite{jpm2006riskmetrics} を組み合わせた共分散推定法を採用する.
共分散推定手法の比較や最適化は本研究の主目的ではないため,以降の数値実験では全ての比較手法に対して同一の共分散推定法を用い,
これを固定した上で,DFL による推定モデル学習の効果に焦点を当てる.

%===================================================================================================
% 第3章:提案手法
%========================
\chapter{提案手法}
%========================
% 二段階最適化モデル
%========================
\section{二段階最適化モデル}
\par
本章では,DFL の枠組みに基づき,平均--分散ポートフォリオ最適化問題を二段階最適化問題として定式化する.
以降,第2章で導入した平均--分散コスト $c(\cdot,\cdot)$($\delta\in[0,1]$ によりリターン項とリスク項を重み付けした形)を用いる.
%========================
% 問題設定と推定モデル
%========================
\subsection*{問題設定と推定モデル}
\par
時点 $i=1,\dots,T$ において,特徴量ベクトル $\boldsymbol{x}_i \in \mathbb{R}^d$ および
実現リターン $\boldsymbol{r}_i \in \mathbb{R}^d$ が観測されるとする.第2章と同様に,
期待リターンの推定値は線形単回帰モデル \eqref{eq:prediction_model} により与えられるものとする.
ここで $\boldsymbol{\theta} \in \mathbb{R}^d$ は学習対象となる回帰係数である.
%========================
% 下位問題:制約付きポートフォリオ最適化
%========================
\subsection*{下位問題:制約付きポートフォリオ最適化}
\par
推定モデル \eqref{eq:prediction_model} に基づく推定値 $\hat{\boldsymbol{r}}_i(\boldsymbol{\theta}, \boldsymbol{x}_i)$ を入力として
得られる各時点 $i$ の投資配分を $\hat{\boldsymbol{w}}_i$ とし,次の制約付き最適化問題の解として定義する.
\begin{equation}
\hat{\boldsymbol{w}}_i(\boldsymbol{\theta}, \boldsymbol{x}_i)
\in
\arg\min_{\boldsymbol{w}_i \in \mathcal{S}}
c\!\left(\boldsymbol{w}_i, \hat{\boldsymbol{r}}_i(\boldsymbol{\theta}, \boldsymbol{x}_i)\right),
\label{eq:lower_level_new}
\end{equation}
\par
ただし,目的関数は次の修正平均--分散コスト関数で与えられる.
\begin{equation}
c(\boldsymbol{w}_i, \boldsymbol{r})
= -(1-\delta)\boldsymbol{r}^\top \boldsymbol{w}_i
+ \frac{\delta}{2}\boldsymbol{w}_i^\top \boldsymbol{V}_i \boldsymbol{w}_i,
\quad 0 \le \delta \le 1.
\label{eq:modified_mvo}
\end{equation}
\par
ポートフォリオの制約集合 $\mathcal{S}$ は
\begin{equation}
\mathcal{S}
= \left\{
\boldsymbol{w}_i \in \mathbb{R}^d
\ \middle|\ 
\boldsymbol{1}^\top \boldsymbol{w}_i = 1,\ 
\boldsymbol{w}_i \ge \boldsymbol{0}
\right\}
\label{eq:feasible_set_new}
\end{equation}
とする.
\par
ここで $\boldsymbol{V}_i \in \mathbb{S}_{++}^d$ は共分散行列であり,$\delta$ はリターン項とリスク項の比重を制御するパラメータである. 
$\delta \to 0$ のときリターン重視,$\delta \to 1$ のときリスク重視の投資配分が得られる.
%========================
% 理想的な投資配分
%========================
\subsection*{理想的な投資配分}
\par
意思決定誤差を評価する基準として,各時点 $i$ において実現リターン $\boldsymbol{r}_i$ が既知であると仮定した場合の理想的な投資配分を次のように定義する.
\begin{equation}
\boldsymbol{w}_i^*
\in
\arg\min_{\boldsymbol{w}_i \in \mathcal{S}}
c(\boldsymbol{w}_i, \boldsymbol{r}_i).
\label{eq:ideal_solution_new}
\end{equation}
\par
この投資配分は実運用では利用できないが,学習ではこの理想解に対する目的関数差を最小化することで意思決定誤差の低減を図る.
%========================
% 上位問題:意思決定誤差最小化
%========================
\subsection*{上位問題:意思決定誤差最小化}
\par
DFL では,予測モデルのパラメータ $\boldsymbol{\theta}$ を,予測誤差ではなく意思決定の質を通じて学習する.本研究では,各時点 $i$ における意思決定誤差を次のように定義する.
\begin{equation}
\ell_i(\boldsymbol{\theta})
= c\!\left(\hat{\boldsymbol{w}}_i(\boldsymbol{\theta}, \boldsymbol{x}_i), \boldsymbol{r}_i\right)
- c(\boldsymbol{w}_i^*, \boldsymbol{r}_i).
\label{eq:decision_loss_new}
\end{equation}
\par
このとき,上位問題は次の二段階最適化問題として定式化される.
\begin{align}
\min_{\boldsymbol{\theta}} \quad
& \frac{1}{T}\sum_{i=1}^{T} \ell_i(\boldsymbol{\theta}),
\label{eq:upper_level_obj}\\
\text{s.t.} \quad
& \hat{\boldsymbol{w}}_i(\boldsymbol{\theta}, \boldsymbol{x}_i)
\ \text{は}\ \eqref{eq:lower_level_new}\ \text{の最適解}.
\label{eq:upper_level_constraint}
\end{align}
\par
すなわち,本研究で扱う問題は,制約付き平均--分散ポートフォリオ最適化を下位問題に含む二段階最適化問題である.
%========================
% 問題の性質と再定式化への動機
%========================
\subsection*{問題の性質と再定式化への動機}
\par
定式化 \eqref{eq:upper_level_obj}--\eqref{eq:upper_level_constraint} は,下位問題に $\arg\min$ 演算子を含むため,直接的な数値計算が困難である.さらに,目的関数は下位問題の最適解写像 $\hat{\boldsymbol{w}}_i(\boldsymbol{\theta},\boldsymbol{x}_i)$ を介して $\boldsymbol{\theta}$ に依存するため,一般に非凸となる.
\par
そこで本研究では,先行研究 \cite{bucarey2024} を参考に,下位問題の最適性条件を用いて $\arg\min$ を含む二段階構造を単一レベルの非凸二次計画問題へ再定式化する.次節では,強双対性条件に基づく定式化(DFL-OPT-D)および KKT 条件に基づく定式化(DFL-OPT-K)を示す.
%========================
% 再定式化
%========================
\section{再定式化}
\par
本節では,前節の方針に従い,単一レベル化した2通りの定式化として,(1) 強双対性条件に基づく DFL-OPT-D,(2) KKT 条件に基づく DFL-OPT-K を示す.
%========================
% 下位問題のラグランジュ関数
%========================
\subsection*{下位問題のラグランジュ関数}
\par
各時点 $i$ における下位問題 \eqref{eq:lower_level_new} を再掲する.
ここでは簡単のため $\hat{\boldsymbol{r}}_i := \hat{\boldsymbol{r}}_i(\boldsymbol{\theta}, \boldsymbol{x}_i)$ とおく.
\begin{align}
\min_{\boldsymbol{w}_i} \quad
& -(1-\delta)\hat{\boldsymbol{r}}_i^\top \boldsymbol{w}_i
+ \frac{\delta}{2}\boldsymbol{w}_i^\top \boldsymbol{V}_i \boldsymbol{w}_i,
\label{eq:lower_level_recall}\\
\text{s.t.} \quad
& \boldsymbol{1}^\top \boldsymbol{w}_i = 1,
\label{eq:lower_level_recall_budget}\\
& \boldsymbol{w}_i \ge \boldsymbol{0}.
\label{eq:lower_level_recall_nonneg}
\end{align}
\par
等式制約および不等式制約に対応するラグランジュ乗数を,それぞれ $\mu_i \in \mathbb{R}$,$\boldsymbol{\lambda}_i \in \mathbb{R}^d_{\ge 0}$ とすると,
ラグランジュ関数は次のように与えられる.
\begin{equation}
\begin{aligned}
\mathcal{L}_i(\boldsymbol{w}_i, \mu_i, \boldsymbol{\lambda}_i)
&=
-(1-\delta)\hat{\boldsymbol{r}}_i^\top \boldsymbol{w}_i
+ \frac{\delta}{2}\boldsymbol{w}_i^\top \boldsymbol{V}_i \boldsymbol{w}_i
+ \mu_i(1-\boldsymbol{1}^\top \boldsymbol{w}_i)
- \boldsymbol{\lambda}_i^\top \boldsymbol{w}_i.
\end{aligned}
\label{eq:lagrangian}
\end{equation}
%========================
% 強双対性条件に基づく再定式化
%========================
\subsection*{強双対性条件に基づく再定式化}
\par
下位問題 \eqref{eq:lower_level_recall} は,$\boldsymbol{V}_i\succ\boldsymbol{0}$ および $\delta>0$ の仮定の下で目的関数が強凸となる凸二次計画問題である.
また,制約集合 $\{\boldsymbol{w}\mid \boldsymbol{1}^\top\boldsymbol{w}=1,\ \boldsymbol{w}\ge\boldsymbol{0}\}$ は非空であり,例えば $\boldsymbol{w}=\frac{1}{d}\boldsymbol{1}$ は等式制約を満たし,かつ $\boldsymbol{w}>\boldsymbol{0}$ を満たす.したがって Slater 条件が成立する.
以上より,本問題は強双対性を満たし,双対問題が存在する.
\par
弱双対性より,任意の主問題の実行可能解 $\boldsymbol{w}_i$ と双対変数 $(\boldsymbol{\lambda}_i,\mu_i)$($\boldsymbol{\lambda}_i\ge\boldsymbol{0}$)に対して,
双対目的値は主問題目的値の下界となる.
\begin{equation}
\begin{aligned}
-(1-\delta)\hat{\boldsymbol{r}}_i^\top \boldsymbol{w}_i
+ \frac{\delta}{2}\boldsymbol{w}_i^\top \boldsymbol{V}_i \boldsymbol{w}_i
\ge
 \mu_i-\frac{\delta}{2}\boldsymbol{w}_i^\top \boldsymbol{V}_i \boldsymbol{w}_i,
\end{aligned}
\label{eq:dual_value}
\end{equation}
\par
強双対性より,最適解においては \eqref{eq:dual_value} の不等式は等号で成立する.
\par
さらに,ラグランジュ関数の一階条件より,
\begin{equation}
\begin{aligned}
\nabla_{\boldsymbol{w}_i}\mathcal{L}_i(\boldsymbol{w}_i,\mu_i,\boldsymbol{\lambda}_i)
&= \delta \boldsymbol{V}_i \boldsymbol{w}_i
-(1-\delta)\hat{\boldsymbol{r}}_i
 - \mu_i \boldsymbol{1}
 - \boldsymbol{\lambda}_i\\
&= \boldsymbol{0}.
\end{aligned}
\label{eq:dual_stationarity}
\end{equation}
\par
また,強双対性と \eqref{eq:dual_stationarity} を用いると,最適値一致条件は次の等式として表せる(導出は Appendix に示す).
\begin{equation}
\delta \boldsymbol{w}_i^\top \boldsymbol{V}_i \boldsymbol{w}_i
-(1-\delta)\hat{\boldsymbol{r}}_i^\top \boldsymbol{w}_i
=\mu_i.
\label{eq:dual_strong_simplified}
\end{equation}
\par
以上を用いることで,二段階最適化問題 \eqref{eq:upper_level_obj}--\eqref{eq:upper_level_constraint} は,次の単一レベル最適化問題として再定式化される.
\begin{align}
\min_{\boldsymbol{\theta}, \{\boldsymbol{w}_i,\mu_i,\boldsymbol{\lambda}_i\}_{i=1}^{T}}
\quad
& \frac{1}{T}\sum_{i=1}^{T}
 \left(
-(1-\delta)\boldsymbol{r}_i^\top \boldsymbol{w}_i
+ \frac{\delta}{2}\boldsymbol{w}_i^\top \boldsymbol{V}_i \boldsymbol{w}_i
\right),
\label{eq:dfl_dual_obj}\\
\text{s.t.} \quad
& \boldsymbol{1}^\top \boldsymbol{w}_i = 1,
\qquad i=1,\dots,T,
\label{eq:dfl_dual_budget}\\
& \boldsymbol{w}_i \ge \boldsymbol{0},
\qquad i=1,\dots,T,
\label{eq:dfl_dual_nonneg}\\
& \boldsymbol{\lambda}_i \ge \boldsymbol{0},
\qquad i=1,\dots,T,
\label{eq:dfl_dual_lambda_nonneg}\\
& \delta \boldsymbol{w}_i^\top \boldsymbol{V}_i \boldsymbol{w}_i
-(1-\delta)\hat{\boldsymbol{r}}_i^\top \boldsymbol{w}_i
= \mu_i,
\qquad i=1,\dots,T,
\label{eq:dfl_dual_value}\\
& \delta \boldsymbol{V}_i \boldsymbol{w}_i
-(1-\delta)\hat{\boldsymbol{r}}_i
- \mu_i \boldsymbol{1}
- \boldsymbol{\lambda}_i
= \boldsymbol{0},
\qquad i=1,\dots,T.
\label{eq:dfl_dual_stationarity}
\end{align}
\par
この定式化を DFL-OPT-D と呼ぶ.
%========================
% KKT 条件に基づく再定式化
%========================
\subsection*{KKT 条件に基づく再定式化}
\par
別の再定式化として,下位問題 \eqref{eq:lower_level_recall} の KKT 条件をすべて制約として組み込む方法を考える.KKT 条件は以下から構成される.
\par
一次の最適性条件
\begin{equation}
\begin{aligned}
\nabla_{\boldsymbol{w}_i}\mathcal{L}_i(\boldsymbol{w}_i,\mu_i,\boldsymbol{\lambda}_i)
&= \delta \boldsymbol{V}_i \boldsymbol{w}_i
-(1-\delta)\hat{\boldsymbol{r}}_i
 - \mu_i \boldsymbol{1}
 - \boldsymbol{\lambda}_i\\
&= \boldsymbol{0},
\end{aligned}
\label{eq:kkt_stationarity}
\end{equation}
\par
実行可能性条件
\begin{align}
\boldsymbol{1}^\top \boldsymbol{w}_i &= 1,
\label{eq:kkt_budget}\\
\boldsymbol{w}_i &\ge \boldsymbol{0},
\label{eq:kkt_w_nonneg}\\
\boldsymbol{\lambda}_i &\ge \boldsymbol{0}.
\label{eq:kkt_lambda_nonneg}
\end{align}
\par
相補性条件
\begin{equation}
\boldsymbol{\lambda}_i \odot \boldsymbol{w}_i = \boldsymbol{0}.
\label{eq:kkt_complementarity}
\end{equation}
\par
これらを用いることで,次の単一レベル最適化問題が得られる.
\begin{align}
\min_{\boldsymbol{\theta}, \{\boldsymbol{w}_i,\mu_i,\boldsymbol{\lambda}_i\}_{i=1}^{T}}
\quad
& \frac{1}{T}\sum_{i=1}^{T}
 \left(
-(1-\delta)\boldsymbol{r}_i^\top \boldsymbol{w}_i
+ \frac{\delta}{2}\boldsymbol{w}_i^\top \boldsymbol{V}_i \boldsymbol{w}_i
\right),
\label{eq:dfl_kkt_obj}\\
\text{s.t.} \quad
& \boldsymbol{1}^\top \boldsymbol{w}_i = 1,
\qquad i=1,\dots,T,
\label{eq:dfl_kkt_budget}\\
& \boldsymbol{w}_i \ge \boldsymbol{0},
\qquad i=1,\dots,T,
\label{eq:dfl_kkt_nonneg}\\
& \boldsymbol{\lambda}_i \ge \boldsymbol{0},
\qquad i=1,\dots,T,
\label{eq:dfl_kkt_lambda_nonneg}\\
& \delta \boldsymbol{V}_i \boldsymbol{w}_i
-(1-\delta)\hat{\boldsymbol{r}}_i
- \mu_i \boldsymbol{1}
- \boldsymbol{\lambda}_i
= \boldsymbol{0},
\qquad i=1,\dots,T,
\label{eq:dfl_kkt_stationarity}\\
& \boldsymbol{\lambda}_i \odot \boldsymbol{w}_i = \boldsymbol{0},
\qquad i=1,\dots,T.
\label{eq:dfl_kkt_complementarity}
\end{align}
\par
この定式化を DFL-OPT-K と呼ぶ.
%========================
% DFL-OPT-D と DFL-OPT-K の関係
%========================
\subsection*{DFL-OPT-D と DFL-OPT-K の関係}
\par
本節では,提案手法である DFL-OPT-D と DFL-OPT-K の理論的関係を整理する.
両者はいずれも同一の下位ポートフォリオ最適化問題 \eqref{eq:lower_level_recall} に基づいて導出されており,
下位問題の最適性条件を異なる形で表現した定式化である.
\par
\begin{proposition}[DFL-OPT-D と DFL-OPT-K の理論的同値性]
\label{prop:equivalence}
各 $i=1,\ldots,T$ に対して,下位のポートフォリオ最適化問題 \eqref{eq:lower_level_recall} が凸二次計画問題であり,Slater 条件を満たすと仮定する.
このとき,DFL-OPT-D(\eqref{eq:dfl_dual_obj}--\eqref{eq:dfl_dual_stationarity})と DFL-OPT-K(\eqref{eq:dfl_kkt_obj}--\eqref{eq:dfl_kkt_complementarity})は同一の解集合を持つ.すなわち,両定式化は理論的に等価である.
\end{proposition}
\par
(証明は Appendix に示す.)
\par
命題 \ref{prop:equivalence} より,DFL-OPT-D と DFL-OPT-K はいずれも下位問題 \eqref{eq:lower_level_recall} の最適性条件を正確に表現しており,理論的には等価であることが分かる.一方で,数値計算の観点からは,両者は異なる性質を持つ.
\par
DFL-OPT-K は,下位問題の KKT 条件を直接組み込んだ定式化であり,制約構造が明示的かつ解釈しやすい.その結果,数値計算において比較的安定した挙動を示し,初期値や探索経路に対する感度も抑えられることが確認された.
一方,DFL-OPT-D は双対関数に基づく非線形等式制約を含むため,数値的な取り扱いが相対的に難しい場合がある.
\par
以上を踏まえ,本研究の数値実験では,理論的同値性を保ちつつ,より安定かつ実装上扱いやすい DFL-OPT-K を主たる提案手法として用い,以降の議論を進める.
DFL-OPT-D は,提案手法の理論的背景を補完する定式化として位置づける.

%========================
% 解写像の局所感度と条件数(追加)
%========================
\subsection*{解写像の局所感度と条件数}
\par
命題 \ref{prop:equivalence} より,DFL-OPT-D と DFL-OPT-K は理論的には同値である.
一方で,第4章の数値実験では,条件数が高い局面において手法間で意思決定誤差の tail や
ポートフォリオの集中度($N_{\mathrm{eff}}$ や $\max_i w_i$)に差が現れることが観察された.
この差を解釈するため,本節では,下位の制約付き凸二次計画問題における最適解写像
$(\hat{\boldsymbol{r}},\boldsymbol{V})\mapsto \boldsymbol{w}^*(\hat{\boldsymbol{r}},\boldsymbol{V})$
の局所感度に関する標準的結果を,本研究の文脈に合わせて整理する.
\par
以下では,各時点を固定し,次の下位問題(平均--分散最適化)を考える:
\begin{equation}
\boldsymbol{w}^*(\hat{\boldsymbol{r}},\boldsymbol{V})
\in
\arg\min_{\boldsymbol{w}}
\left\{
-(1-\delta)\hat{\boldsymbol{r}}^\top \boldsymbol{w}
+\frac{\delta}{2}\boldsymbol{w}^\top \boldsymbol{V}\boldsymbol{w}
\ \middle|\
\boldsymbol{1}^\top \boldsymbol{w}=1,\ \boldsymbol{w}\ge\boldsymbol{0}
\right\}.
\label{eq:lower_qp_sensitivity}
\end{equation}
\par
本問題は $\delta>0$ かつ $\boldsymbol{V}\succ\boldsymbol{0}$ の下で強凸であり,最適解は一意に定まる.
さらに,不等式制約のアクティブ集合($\boldsymbol{w}$ の正の成分集合)が近傍で不変である局所では,
$\boldsymbol{w}^*$ は $\hat{\boldsymbol{r}}$ に関して滑らかに変化し,その感度は
$\boldsymbol{V}$ の数値的性質(特に最小固有値)によって上から抑えられることが知られている
(パラメトリック凸QPの感度解析における標準結果).
\par
この点を明確化するため,以下の命題を示す.
(厳密な導出は Appendix に示す.)
本命題はパラメトリック凸QPの感度解析における標準結果であり,本研究では第4章の実験結果の解釈のために再掲する。
また、大域的なLipschitz性ではなく,アクティブ集合固定下の局所結果に限定する。

\begin{proposition}[最適解写像の局所感度(アクティブ集合固定)]
\label{prop:local_sensitivity}
$\delta>0$,$\boldsymbol{V}\succ\boldsymbol{0}$ とし,
\eqref{eq:lower_qp_sensitivity} の最適解を $\boldsymbol{w}^*(\hat{\boldsymbol{r}},\boldsymbol{V})$ とする.
ある $(\hat{\boldsymbol{r}},\boldsymbol{V})$ の近傍で,
アクティブ集合 $A:=\{j\mid w^*_j>0\}$ が不変であると仮定する.
このとき,$A$ 上の解 $\boldsymbol{w}^*_A$ は $\hat{\boldsymbol{r}}_A$ に関して affine であり,
そのヤコビアンは
\begin{equation}
\frac{\partial \boldsymbol{w}^*_A}{\partial \hat{\boldsymbol{r}}_A}
=
\frac{1-\delta}{\delta}\left(
\boldsymbol{V}_{AA}^{-1}
-
\boldsymbol{V}_{AA}^{-1}\boldsymbol{1}
(\boldsymbol{1}^\top \boldsymbol{V}_{AA}^{-1}\boldsymbol{1})^{-1}
\boldsymbol{1}^\top \boldsymbol{V}_{AA}^{-1}
\right)
\label{eq:jacobian_affine}
\end{equation}
で与えられる.
したがって,同一アクティブ集合内での任意の摂動 $\Delta\hat{\boldsymbol{r}}$ に対し
\begin{equation}
\|\Delta \boldsymbol{w}^*\|_2
\le
C_A\,
\frac{1-\delta}{\delta}\,
\frac{1}{\lambda_{\min}(\boldsymbol{V}_{AA})}\,
\|\Delta \hat{\boldsymbol{r}}\|_2
\label{eq:lipschitz_bound}
\end{equation}
を満たす定数 $C_A<\infty$ が存在する.
\end{proposition}

\par
命題 \ref{prop:local_sensitivity} は,アクティブ集合が固定される局所では,
$\hat{\boldsymbol{r}}$ の誤差(あるいは数値誤差)が $\boldsymbol{w}^*$ に増幅される度合いが
$\lambda_{\min}(\boldsymbol{V}_{AA})$ により支配されることを意味する.
特に,$\boldsymbol{V}$ の条件数が大きい局面では一般に
$\lambda_{\min}(\boldsymbol{V}_{AA})$ が小さくなりやすく,
同じ大きさの $\Delta\hat{\boldsymbol{r}}$ に対して $\boldsymbol{w}^*$ が大きく変動し得る.
さらに,アクティブ集合の切替えが生じる境界近傍では,
この変動が「少数資産への極端集中」へ遷移する形で観測される可能性がある.
\par
本研究では共分散推定として OAS 縮小を用いるため,
$\boldsymbol{V}_\phi=(1-\phi)\boldsymbol{S}+\phi \alpha \boldsymbol{I}$ と書ける.
このとき
$\lambda_{\min}(\boldsymbol{V}_\phi)\ge (1-\phi)\lambda_{\min}(\boldsymbol{S})+\phi\alpha$
より,$\lambda_{\min}$ が底上げされ,
\eqref{eq:lipschitz_bound} の上界(すなわち解写像の感度)が抑制される方向に働く.
第4章では,この感度の観点と整合する形で,
条件数レジーム別に意思決定誤差の tail および $N_{\mathrm{eff}}$ や極端集中確率を比較する.

%==============================================================================================
% 第4章:数値実験
%========================
\chapter{数値実験}
\par
\noindent
本章では,第3章で提案した DFL-OPT-D および DFL-OPT-K の数値的挙動と実務的有効性を,実データを用いたタクティカル・アセット・アロケーション(Tactical Asset Allocation; TAA)の文脈で評価する.特に,(i)dual 定式化と KKT 定式化の数値的差異,(ii)初期化の影響,(iii)提案手法がどのような状況で相対的な優位性を示すか,という点に焦点を当てる.
%========================
% 実データ実験の設定
%========================
\section{実データ実験の設定}
\par
\noindent
本節では,実験の基本方針,使用データ,共分散推定方法,比較手法,ならびに初期化およびソルバー設定について述べる.本章を通じて,比較手法間で実験設定を可能な限り統一し,学習枠組みおよび定式化の違いに起因する差異に焦点を当てる.
%========================
% 実験の基本方針
%========================
\subsection*{実験の基本方針}
\par
第3章で示したとおり, DFL-OPT-D および DFL-OPT-K はいずれも,下位に制約付き平均--分散最適化問題を含む非凸な非線形最適化問題として定式化される.そのため,数値計算においては初期解やソルバー設定に依存して,異なる局所解へ収束する可能性がある.
\par
本研究の目的は,特定の初期化やハイパーパラメータ調整によって得られた「最良の局所解」を主張することではない.むしろ,実務的に自然な制約条件と統一された実験設定の下で, DFL に基づく定式化がどのような挙動・特性を示すかを体系的に評価することを目的とする.
\par
このため,本章では可能な限り
\begin{itemize}
  \item 使用データ
  \item 特徴量設計
  \item 制約条件
  \item 評価指標
  \item 初期化およびソルバー設定
\end{itemize}
を比較手法間で統一し,モデル構造および学習方式の違いそのものに起因する差異に焦点を当てる.
%========================
% 使用データおよび学習・再バランス設定
%========================
\subsection*{使用データおよび学習・再バランス設定}
\par
実データ実験では,短期タクティカル・アセット・アロケーション(Tactical Asset Allocation; TAA)を想定し, Yahoo Finance から取得した調整後終値に基づく週次リターン(2006 年 1 月から 2025 年 12 月)を用いる.投資対象は,異なるリスク特性を代表する ETF として,以下の 4 資産を選択する.
\begin{itemize}
  \item SPDR S\&P 500 ETF Trust(SPY):米国株式
  \item SPDR Gold Shares(GLD):金
  \item iShares MSCI Emerging Markets ETF(EEM):新興国株式
  \item iShares 20+ Year Treasury Bond ETF(TLT):米国長期国債
\end{itemize}
\par
各時点 $t$ における特徴量 $\boldsymbol{x}_t$ として,直近 26 週の週次リターン平均を用いる.これは,短期的なトレンド情報を反映しつつ,週次データに内在するノイズを一定程度平滑化するための設定である.
\par
モデルの学習およびポートフォリオの更新はローリング手順に基づき,直近 26 週(約半年)の週次リターンを用いて推定を行い,推定したパラメータは次の 4 週間(約 1 か月)にわたって固定して用いる.これにより,週次データの短期ノイズに過度に反応することを抑えつつ,半年程度の情報に基づく短期的な状態推定と月次の運用更新という時間スケールを両立する.
\par
なお,以降の実験ではリスクとリターンのトレードオフ係数を中庸的な設定として $\delta=0.5$ に固定する.本章ではハイパーパラメータ最適化は目的とせず,すべての手法で学習窓長と更新頻度を共通に固定することで,第 3 章の定式化および学習枠組みの差異の影響を比較する.
%========================
% 共分散行列の推定
%========================
\subsection*{共分散行列の推定}
\par
本研究では,ポートフォリオ最適化に用いる共分散行列 $\boldsymbol{V}_t$ を,時間減衰を考慮した標本共分散行列に対して Oracle Approximating Shrinkage(OAS)を適用する方法により推定する \cite{chen2010oas}.
\par
まず,時点 $t$ における EWMA 共分散行列 $\boldsymbol{S}_t$ を次式で定義する.
\begin{equation}
\boldsymbol{S}_t
= (1-\alpha)\sum_{k=0}^{L-1}
\alpha^k
(\boldsymbol{r}_{t-k} - \bar{\boldsymbol{r}}_t)
(\boldsymbol{r}_{t-k} - \bar{\boldsymbol{r}}_t)^\top,
\label{eq:ewma_cov}
\end{equation}
\par
ここで,$L=13$ はローリング窓長(約四半期)であり,本研究の短期ローリング設定に基づき固定する.また,$\alpha=0.97$ は時間減衰率であり,
実務における業界慣例に基づき固定する \cite{jpm2006riskmetrics}.$\bar{\boldsymbol{r}}_t$ は同窓内の平均リターンを表す.
\par
次に,OAS に基づく縮小共分散行列 $\boldsymbol{V}_t$ を次のように定義する.
\begin{equation}
\boldsymbol{V}_t
= (1-\phi_t)\boldsymbol{S}_t
+ \phi_t
\frac{\mathrm{tr}(\boldsymbol{S}_t)}{d}
\boldsymbol{I},
\label{eq:oas_cov}
\end{equation}
\par
ここで $\phi_t\in[0,1]$ は縮小強度であり,OAS により解析的に決定される.また,EWMA のように観測に重みを付ける場合,$n_{\mathrm{eff}}$ は「時間減衰率 $\alpha$ に対応する実効サンプルサイズ」とみなせる \cite{jpm2006riskmetrics}.OAS は,小標本下において標本共分散行列の推定誤差を抑制しつつ,分散構造の情報を保持する点で知られており,本研究のような短期ローリング設定と高い親和性を持つ \cite{chen2010oas}.縮小係数 $\phi_t$ および $n_{\mathrm{eff}}$ の具体式は Appendix に示す.
\par
本研究では,共分散推定手法の比較や最適化は目的とせず,すべての手法に対して同一の $\boldsymbol{V}_t$ を用いる.これにより,推定リターンと意思決定の統合方法(PFL / IPO / DFL)の違いに焦点を当てる.
%========================
% 評価指標
%========================
\subsection*{評価指標}
\par
本章の数値実験におけるポートフォリオ性能の評価には,リターン水準,リスク水準,リスク調整後パフォーマンス,および売買行動の安定性を多面的に捉えるため,以下の指標を用いる.いずれの指標も,取引コスト(bps)を考慮した実行ベースの損益系列に基づいて算出する.
\par
\noindent リターン指標
\begin{itemize}
  \item 年率リターン(Annualized Return):期間全体の累積リターンから年率換算した平均成長率を用いる.
  \item 最終資産価値(Final Wealth):初期資産を 1 としたときの評価期間終了時点における累積資産額を用いる.
\end{itemize}
\par
\noindent リスク調整後指標
\begin{itemize}
  \item Sharpe 比:無リスク利子率を 0 と仮定し,年率リターンを年率ボラティリティで除した値として定義する.
    \[
    \mathrm{Sharpe}=\frac{\mu}{\sigma},
    \]
    ここで $\mu$ は年率リターン,$\sigma$ は年率ボラティリティである.
\end{itemize}
\par
\noindent リスク指標
\begin{itemize}
  \item 年率ボラティリティ(Annualized Volatility):週次リターンの標準偏差を年率換算した値とする.
  \item 最大ドローダウン(Maximum Drawdown):累積資産曲線におけるピークからの最大下落率を測定する.
  \item CVaR$_{95}$:リターン分布の下位 5\% における平均損失を用い,極端な損失リスクを評価する.
\end{itemize}
\par
\noindent 売買行動・安定性指標
\begin{itemize}
  \item 平均ターンオーバー(Turnover):各リバランス時点におけるポートフォリオ変更量の平均値を用いる.
  \item スイッチ頻度(Switch Frequency):リバランス時点において,最大ウェイトを持つ資産が前期から変更された割合を測定する.本指標は,意思決定の安定性およびポートフォリオ構成の急激な変化の度合いを捉える目的で導入する.
\end{itemize}
\par
以上の評価指標は,すべての比較手法に対して同一の定義および算出方法を適用する.これにより,手法間の性能差が定式化および学習枠組みの違いに起因するものであることを明確にする.

%========================
% 比較手法
%========================
\subsection*{比較手法}
\par
実データ実験では,予測モデルや制約条件等を可能な限り統一したうえで,以下の手法を比較対象として用いる.
\par
\noindent 提案手法
\begin{itemize}
  \item DFL-OPT-D:第3章で導出した強双対性条件に基づく提案手法.
  \item DFL-OPT-K:第3章で導出した KKT 条件に基づく提案手法.
\end{itemize}
\par
\noindent Prediction-Focused Learning の基準
\begin{itemize}
  \item OLS + MVO:期待リターンを最小二乗誤差で推定し,得られた推定値 $\hat{\boldsymbol{r}}$ を平均--分散最適化に入力して配分を決定する,実務でも標準的な構成.これを PFL の基準手法として採用する.
\end{itemize}
\par
\noindent 予測統合型手法
\begin{itemize}
  \item IPO-GRAD:最適化問題を学習ループに組み込み,下位最適化問題を通じて勾配を伝播させる End-to-End 手法 \cite{butlerkwon2023}.
\end{itemize}
\par
\noindent Decision-Focused Learning 系
\begin{itemize}
  \item DFL-CF:\eqref{eq:upper_level_obj}--\eqref{eq:upper_level_constraint} の下位問題において
  制約を外した場合に得られる解析解に基づく手法であり,\cite{butlerkwon2023} に示される解析的解に対応する.
  本研究では,解析解ベースの基準および初期化候補として用いる.
  \item SPO+:線形最適化問題に対して機会損失の凸上界を構成することで,勾配ベースの学習を可能にした手法 \cite{elmachtoubgrigas2022}.
  本研究の制約付き平均--分散問題には直接適用できないため,近似的設定により比較対象に含める.
\end{itemize}
\par
\noindent ベンチマーク(運用戦略)
\begin{itemize}
  \item Buy-and-Hold(S\&P 500):株式市場に対する単純な長期保有戦略.
  \item 等分散投資($1/N$):推定誤差に依存しない頑健な基準配分.
\end{itemize}

%========================
% 初期化およびソルバー設定
%========================
\subsection*{初期化およびソルバー設定}
\par
第3章で述べたとおり,本研究で扱う最適化問題は非凸であり,初期化により到達する局所解が変化し得る.
本研究では,非線形最適化ソルバーとして KNITRO を用い,ソルバー設定は原則として全手法で統一する.
\par
DFL 系手法では,実データ実験におけるデフォルト初期値として $\boldsymbol{\theta}=\boldsymbol{0}$ を用い,
その他の変数は中立的な初期値から開始する.後続の実験では,DFL-CF に基づく初期化も導入し,初期化の影響を検証する.

\setcounter{savedtocdepth}{\value{tocdepth}}
\setcounter{tocdepth}{2}

%========================
% 実験結果
%========================
\section{実験結果}
\addtocontents{toc}{\protect\setcounter{tocdepth}{2}}

%========================
% 実データによるベースライン比較
%========================
\subsection{実データによるベースライン比較}

本節では,第3章で導出した提案手法である DFL-OPT-K について,
実データを用いた比較手法との結果を示す.
以降の数値実験では,DFL-OPT-K を主たる提案手法として扱い,
下位問題の初期解には解析解に基づく DFL-CF を用いる.まず,図\ref{fig:baseline_cumreturn}に,
2006年1月から2025年12月までの全期間における各手法の累積資産推移を示す.

\begin{figure}[H]
\centering
\includegraphics[width=0.95\linewidth]{figs/baseline_cum_return.png}
\caption{累積資産推移}
\label{fig:baseline_cumreturn}
\end{figure}

図\ref{fig:baseline_cumreturn}より,DFL-OPT-K は比較手法と比べて,長期運用期間を通じて相対的に安定した資産成長を示していることが確認できる.
特に,大きな下落局面においても資産の毀損が抑制されており,時系列的に見て過度な変動を伴わない挙動が観察される.
\par
次に,表\ref{tab:baseline_summary}に主要な評価指標をまとめる.(太字:1位, 下線:2位. 提案手法は網掛け.)DFL-OPT-K は Sharpe 比,最終資産および年率リターンにおいて最も高い値を示しており,
年率ボラティリティや CVaR$_{95}$ といったリスク指標においても,等分散投資に次ぐ水準に留まっている.
これらの結果は,図\ref{fig:baseline_cumreturn}で観察された安定した資産推移が,リターン水準とリスク指標の双方において定量的にも裏付けられていることを示している.

\begin{table}[H]
\centering
\caption{評価指標の比較}
\label{tab:baseline_summary}
\small
\setlength{\tabcolsep}{4pt}
\begin{tabular}{lrrrrrr}
\hline
Model & Sharpe & Terminal & Ann.\ Return & Ann.\ Vol & CVaR$_{95}$ \\
\hline
\rowcolor{gray!10}
DFL-OPT-K     & \textbf{0.81} & \textbf{9.14} & \textbf{12.27} & \underline{15.16} & \underline{4.94} \\
DFL-CF        & 0.61          & 6.07          & 10.57          & 17.36          & 5.65 \\
IPO-GRAD      & \underline{0.71} & \underline{8.42} & \underline{12.19} & 17.23          & 5.47 \\
SPO+          & 0.60          & 6.07          & 10.66          & 17.91          & 5.73 \\
PFL           & 0.34          & 2.49          &  6.42          & 19.04          & 6.68 \\
Buy\&Hold     & 0.57          & 5.64          & 10.35          & 18.14          & 6.19 \\
1/N           & 0.62          & 3.64          &  7.13          & \textbf{11.42} & \textbf{3.65} \\
\hline
\end{tabular}
\end{table}

以上より,DFL-OPT-K は実データにおいて,
既存手法と比較して競争力のある投資性能を示していることが確認できる.
次節以降では,
DFL-OPT-K の数値的性質や探索挙動に焦点を当て,
初期解の役割や最適化過程の安定性について詳細に検討する.

%========================
% 探索経路制御の効果
%========================
\subsection{探索経路制御の効果}
\par
本節では,提案手法 DFL-OPT-K において,探索経路の制御が数値的安定性および意思決定品質に与える影響を検証する.
第3章で述べたとおり,DFL-OPT-K は非凸な最適化問題を含むため,初期化や正則化といった探索経路の設計が,
収束挙動および得られる解の安定性に影響を与える可能性がある.
ここでは,解析解に基づく初期解の導入およびパラメータに対する罰則項の付加という二つの探索経路制御手法に着目する.

\subsubsection*{正則化強度 $\eta$ の選択}
\par
$\theta$ が解析解 $\theta_{\mathrm{CF}}$ から過度に乖離することを防ぐため,
本研究では目的関数に $\ell_2$ 型の罰則項
$\eta \lVert \theta - \theta_{\mathrm{CF}} \rVert_2^2$
を導入する.
正則化強度 $\eta$ は,データ全体(2006/01/01--2025/12/31)を
前半 10 年(2006/01/01--2015/12/31)と後半 10 年(2016/01/01--2025/12/31)に分割し,
前半期間のみを用いて選択した.
後半期間は,最終的な性能評価のために保持し,$\eta$ の選択には一切用いていない.

\par
前半 10 年の検証では,時系列の順序を保つため,学習期間を過去から順次延長し,
将来側の区間を検証に用いるホールドアウト検証を行った.
具体的には,学習期間を 2006 年から段階的に拡張し,
各段階でそれ以降の期間を検証区間とする複数の分割
(例:2006--2007 を学習,2008--2015 を検証;$\ldots$;2006--2012 を学習,2013--2015 を検証)を構成した.
各分割において,候補集合
$\eta \in \{0, 0.01, 0.05, 0.1, 0.5, 1, 5, 10, 50, 100, 500\}$
に対する検証性能を算出し,分割間で集計した指標に基づいて比較を行った.

\par
$\eta$ の選択基準としては,検証区間における Sharpe 比の平均値を主指標とし,
同程度の Sharpe 比を示す候補が複数存在する場合には,
下方リスクの観点から CVaR$_{95}$ が小さいものを優先した.
その結果,$\eta=5$ が前半期間の検証において最も安定して良好な性能を示したため,
以降の実験ではこの値を採用する.

\par
\begin{figure}[H]
\centering
\begin{subfigure}[t]{0.49\linewidth}
\centering
\includegraphics[width=\linewidth]{figs/lambda_tuning_sharpe_boxplot.png}
\caption{Sharpe 比}
\label{fig:lambda_tuning_sharpe_boxplot}
\end{subfigure}
\hfill
\begin{subfigure}[t]{0.49\linewidth}
\centering
\includegraphics[width=\linewidth]{figs/lambda_tuning_cvar95_boxplot.png}
\caption{CVaR$_{95}$}
\label{fig:lambda_tuning_cvar95_boxplot}
\end{subfigure}
\caption{検証期間における正則化強度ごとの性能分布}
\label{fig:lambda_tuning_boxplots}
\end{figure}

\subsubsection*{投資成績への影響}
\par
最後に,探索経路制御が実際の投資成績に与える影響を確認する.
ここでは,初期解および罰則項を導入した DFL-OPT-K を用いて,
時系列ホールドアウトの評価期間(2016/01/01--2025/12/31)における運用結果を確認する.
評価指標は取引コストを考慮した実行ベースの損益系列に基づいて算出する.
\par
\par
\begin{figure}[H]
\centering
\includegraphics[width=0.95\linewidth]{figs/holdout_cum_wealth_2016_2025.png}
\caption{ホールドアウト期間における累積資産推移}
\label{fig:holdout_cum_wealth_2016_2025}
\end{figure}
\par
\par
図\ref{fig:holdout_cum_wealth_2016_2025}に示す通り,
解析解初期化+罰則項はホールドアウト期間において
他の設定と比較して安定した資産成長を示している.
実際,Sharpe 比は
初期化($\theta=0$)(0.59),
解析解初期化(0.61)に対し,
解析解初期化+罰則項では 0.67 と最も高い値を示した.
\par
\par
これらの結果から,
探索経路制御は数値的安定性を向上させるだけでなく,
投資成績を大きく損なうことなく,
場合によってはリスク調整後指標の改善にも寄与していることが確認できる.

%========================
% 提案手法の性質分析
%========================
\subsection{提案手法の性質分析}
\par
本節では,前節までで示した提案手法(DFL-OPT-K)の実証的な性能差が,どのような性質に起因しているのかを詳しく分析する.
特に本研究では,「予測精度の高さ」そのものではなく,「予測誤差を含んだ状況下で,どのような意思決定が行われているか」という観点から,提案手法の挙動を整理する.
\par
以下では,(i) 予測精度と意思決定品質の乖離,(ii) 問題条件の難易度(条件数)に応じた挙動の変化,(iii) ポートフォリオ解 $\boldsymbol{w}$ の構造的特徴,
および (iv) 危機局面における挙動,の順に検討する.

\subsubsection{予測精度と意思決定品質の乖離}
\par
まず,予測モデルの精度と,それに基づく投資成果との関係を確認する.


\begin{figure}[H]
\centering
\begin{subfigure}[t]{0.48\linewidth}
  \centering
  \includegraphics[width=\linewidth]{figs/r2_vs_sharpe_scatter.png}
  \caption{$R^2$ と Sharpe 比}
  \label{fig:r2_vs_sharpe}
\end{subfigure}
\hfill
\begin{subfigure}[t]{0.48\linewidth}
  \centering
  \includegraphics[width=\linewidth]{figs/r2_vs_cvar_scatter.png}
  \caption{$R^2$ と CVaR$_{95}$}
  \label{fig:r2_vs_cvar95}
\end{subfigure}
\caption{決定係数 $R^2$ と意思決定品質の関係}
\label{fig:r2_vs_decision_quality}
\end{figure}

\par
図\ref{fig:r2_vs_decision_quality}は,各手法について,予測精度(決定係数 $R^2$)と意思決定品質(投資成績)との関係を示したものである.
\par
これらの図から,予測精度が必ずしも高い手法が高い Sharpe 比や低い CVaR を達成しているわけではないことが確認できる.
特に,PFL 系手法では比較的高い予測精度を示す場合であっても,投資成績が必ずしも改善されないケースが見られる.
\par
一方,提案手法 DFL-OPT-K は,予測精度が中程度である場合でも,相対的に高い Sharpe 比や良好な下方リスク指標を示している.
これは,提案手法が予測誤差を含んだ状況においても,最終的な意思決定を重視した学習・最適化を行っていることを示唆している.

\subsubsection{条件数レジーム別の比較分析}
\par
次に,下位最適化問題の数値的な難易度が,意思決定の品質にどのような影響を与えるかを分析する.
本研究では,各時点における共分散行列の条件数に基づき,市場環境を「低・中・高」の3つの条件数レジームに分類する.

\begin{table}[H]
\centering
\caption{条件数レジーム別: 意思決定誤差と有効資産数 $N_{\mathrm{eff}}$}
\label{tab:cond_regime_decision_error_neff}
{\footnotesize
\renewcommand{\arraystretch}{1.1}
\setlength{\tabcolsep}{3pt}
\begin{tabular}{lrrrr|rrrr|rrrr}
\hline
 & \multicolumn{4}{c|}{低(条件数)} & \multicolumn{4}{c|}{中(条件数)} & \multicolumn{4}{c}{高(条件数)} \\
\cline{2-13}
Model & 平均 & p90 & p95 & $N_{\mathrm{eff}}$ & 平均 & p90 & p95 & $N_{\mathrm{eff}}$ & 平均 & p90 & p95 & $N_{\mathrm{eff}}$ \\
\hline
\rowcolor{gray!10}
DFL-OPT-K & 84.91 & \textbf{201.1} & \textbf{249.1} & \textbf{1.420} & \textbf{94.60} & \textbf{217.6} & \textbf{286.4} & \textbf{1.482} & 120.9 & \textbf{263.7} & \textbf{360.7} & \textbf{1.530} \\
DFL-CF    & \textbf{83.56} & 223.7 & 256.7 & 1.062 & 100.1 & 258.6 & 348.2 & 1.059 & 122.9 & 309.9 & 405.7 & 1.049 \\
IPO-GRAD  & \underline{83.98} & \underline{212.7} & \underline{255.6} & 1.038 & \underline{98.92} & \underline{243.9} & 330.6 & 1.070 & \underline{120.6} & \underline{293.2} & \underline{392.2} & 1.066 \\
SPO+      & 85.08 & 218.2 & 256.2 & 1.036 & 101.6 & 258.2 & \underline{310.6} & 1.049 & \textbf{119.7} & 320.7 & 410.9 & 1.071 \\
PFL       & 85.48 & 218.2 & 263.9 & \underline{1.079} & 101.1 & \underline{257.5} & 336.0 & \underline{1.101} & 132.0 & 329.1 & 481.3 & \underline{1.106} \\
\hline
\end{tabular}
}
\end{table}

\par
表\ref{tab:cond_regime_decision_error_neff}には,
各レジームにおける意思決定誤差および有効資産数 $N_{\mathrm{eff}}$をまとめている.
\par
この結果から,条件数が高くなるにつれて,多くの比較手法では意思決定誤差が大きくなり,有効資産数が急激に減少する傾向が確認される.
特に,条件数が高いレジームでは,解が一部の資産に極端に集中する傾向が顕著である.
\par
一方,提案手法 DFL-OPT-K は,高条件数レジームにおいても意思決定誤差の増大が相対的に抑制されており,有効資産数も安定して維持されている.
これは,提案手法が数値的に不安定な状況においても,意思決定を過度に歪めない構造を持っていることを示唆する.

\par
この挙動は,制約付き凸二次計画問題における最適解が,
アクティブ制約集合に応じた \emph{piecewise affine} な構造を持つことと整合的である.
すなわち,共分散行列の条件数が高い局面では,
わずかな予測誤差や数値誤差がアクティブ集合の切り替えを誘発し,
解が一部の資産に極端に集中するレジームへ遷移しやすい.
提案手法 DFL-OPT-K では,
このようなレジーム切り替えが相対的に抑制されている可能性が示唆される.
さらに,このような高条件数レジームにおける意思決定誤差の tail 抑制は,実際の投資成果にも部分的に反映されている.
表4.2に示すように,高条件数レジームでは多くの手法で意思決定誤差(平均・tail)が増大する一方,提案手法 DFL-OPT-K は p90/p95 の観点で相対的に小さい値を示しており,数値的に不安定な局面において極端な意思決定の失敗を回避しやすい可能性が示唆される.


\subsubsection{ポートフォリオ解 $\boldsymbol{w}$ の構造的特徴}
\par
前節では,共分散行列の条件数に応じて,
意思決定誤差や解の安定性が大きく変化することを確認した.
本節では,そのような数値的特性の違いが,
実際にどのようなポートフォリオ構造として現れているかを分析する.


\begin{table}[H]
\centering
\caption{ポートフォリオ構造の要約(太字:1位, 下線:2位. 提案手法は網掛け.)}
\label{tab:portfolio_structure_summary}
\footnotesize
\setlength{\tabcolsep}{5pt}
\begin{tabular}{lrrrrr}
\hline
Model & HHI(平均) & $P(\max_i w_i \ge 0.95)$ & 捕捉率(\%) & 平均ターンオーバー(\%) & 平均 $N_{\mathrm{eff}}$ \\
\hline
\rowcolor{gray!10}
DFL-OPT-K & \textbf{0.772} & \textbf{0.490} & \textbf{43.1} & 24.01 & \textbf{1.477} \\
DFL-CF    & 0.968 & 0.915 & 27.2 & 24.17 & 1.057 \\
IPO-GRAD  & 0.967 & 0.912 & 27.3 & 25.24 & 1.058 \\
SPO+      & 0.970 & 0.915 & 30.1 & \underline{21.02} & 1.052 \\
PFL       & \underline{0.947} & \underline{0.866} & \underline{30.6} & \textbf{20.73} & \underline{1.095} \\
\hline
\end{tabular}
\end{table}
\par
表\ref{tab:portfolio_structure_summary}には,各手法について,以下をまとめている.

\par
\begin{itemize}
  \item 平均ウェイト
  \item $P(\max_i w_i \ge 0.95)$(極端集中の発生確率)
  \item 週次リターン最大資産の捕捉率
  \item 平均ターンオーバーおよびスイッチ頻度
\end{itemize}
\par
この結果から,PFL 系手法では,
特定の資産への極端な集中や高頻度なポジション切替が生じやすいことが確認される.
これは,意思決定問題の解が,
異なるアクティブ集合に対応するレジーム間を頻繁に行き来している可能性を示唆する.
これに対し,提案手法 DFL-OPT-K は,
極端な集中の発生確率を低く抑えつつ,
必要な局面では資産配分を調整するという,
より滑らかなポートフォリオ更新を行っている.
\par
このような構造的特徴は,前節で確認した下方リスクの抑制や,安定した Sharpe 比の達成と整合的である.

\subsubsection{危機局面における挙動}
\par
最後に,市場の急変動が生じた危機局面における挙動を検証する.
本研究では,リーマンショック期,COVID-19 初期局面,およびインフレ局面を代表的な危機局面として取り上げる.
\par

\begin{table}[H]
\centering
\caption{危機局面における性能指標(太字:1位, 下線:2位. 提案手法は網掛け.)}
\label{tab:crisis_event_metrics}
\footnotesize
\setlength{\tabcolsep}{4pt}
\begin{tabular}{lrr|rr|rr}
\hline
 & \multicolumn{2}{c|}{リーマン危機} & \multicolumn{2}{c|}{コロナショック} & \multicolumn{2}{c}{インフレ局面} \\
\cline{2-7}
Model & Sharpe & CVaR$_{95}$ & Sharpe & CVaR$_{95}$ & Sharpe & CVaR$_{95}$ \\
\hline
\rowcolor{gray!10}
DFL-OPT-K & 0.238          & \textbf{6.87} & \textbf{1.047} & \textbf{8.13} & \textbf{0.615} & \textbf{3.89} \\
DFL-CF    & \textbf{0.493} & \underline{7.42}          & \underline{0.996}          & \underline{9.42}          & $-$0.200       & \underline{4.51} \\
IPO-GRAD  & $-$0.168       & 9.76          & $-$0.210       & 12.11         & 0.021          & 4.76 \\
SPO+      & \underline{0.249}          & 7.77          & 0.010          & 11.32         & \underline{0.123}          & 4.85 \\
PFL       & $-$0.385       & 9.88          & $-$0.291       & 12.06         & $-$0.402       & 4.91 \\
\hline
\end{tabular}
\end{table}

\par
ここには,各局面における Sharpe 比,下方リスク指標(CVaR または MaxDD),およびターンオーバーをまとめている.
\par
危機局面において,提案手法は相対的に下方リスクを抑制しつつ,過度な取引を伴わない安定した挙動を示している.
特に,急激な市場下落局面においても,ポートフォリオが極端な集中状態に陥ることを回避している点が特徴的である.

%========================================================================================
% 第5章:結論
%========================
\chapter{結論}
\par
本章では,本研究で得られた知見を総括する.方法論的には,
制約付き平均--分散ポートフォリオ最適化を下位問題として含む DFL を悲観的二段階最適化として定式化し,
下位問題の最適性条件に基づく2つの単一レベル再定式化(DFL-OPT-D,DFL-OPT-K)を導出した.また,理論的同値性と数値計算上の挙動が必ずしも一致しない点を,
実データ実験で検証可能な形で整理した.実証的には,週次TAAの実データ実験において,提案手法(DFL-OPT-K)が $R^2$ の向上に依らずに, 
Sharpe 比の改善や CVaR$_{95}$ の低下を通じて意思決定品質を向上させ得ることを示した.
改善は平均だけでなく分布の裾(p90/p99)の抑制としても現れ,条件数が高い局面で相対的に下方リスクを抑える傾向が確認された.
また,この改善は売買量や切替頻度の増加だけでは説明されにくく,COVID-19 局面では安全資産(TLT)への動的シフトが観察された.
今後は,非凸性に伴う局所解・初期化依存の低減,多資産化と現実的制約の導入,および感度分析や解法安定化を通じた実務適用可能性の検証を進める.

%========================
% 謝辞
%========================
\newpage
\chapter*{謝辞}
\addcontentsline{toc}{chapter}{謝辞}
本研究をご指導くださった高野祐一准教授をはじめ,議論に協力してくださった研究室の皆様に深く感謝いたします.

%========================
% 参考文献
%========================
\newpage
\bibliographystyle{jplain}
\bibliography{cite}

%========================
% Appendix
%========================
\newpage
\appendix
\chapter*{Appendix}
\addcontentsline{toc}{chapter}{Appendix}
\section*{A. 命題 \ref{prop:equivalence} の証明}
\addcontentsline{toc}{section}{A. 命題 \ref{prop:equivalence} の証明}
\par
本章では,DFL-OPT-D と DFL-OPT-K が理論的に同値であることを示す.
\begin{proof}
各 $i=1,\dots,T$ に対して,下位のポートフォリオ最適化問題 \eqref{eq:lower_level_recall} は目的関数が強凸な二次関数であり,
等式制約および非負制約からなる凸制約集合を持つ凸二次計画問題である.また,$\boldsymbol{w}_i>\boldsymbol{0}$ かつ $\boldsymbol{1}^\top\boldsymbol{w}_i=1$ を満たす
内点が存在するため,Slater 条件が成り立つ.したがって,強双対性が成立する.
\par
強双対性の成立より,下位問題の最適解は KKT 条件を満たす点と必要十分に一致する.すなわち,ある双対変数 $(\mu_i,\boldsymbol{\lambda}_i)$ が存在して,停留条件
\[
\delta \boldsymbol{V}_i \boldsymbol{w}_i-(1-\delta)\hat{\boldsymbol{r}}_i-\mu_i\boldsymbol{1}-\boldsymbol{\lambda}_i=\boldsymbol{0},
\]
主問題の可行性
\[
\boldsymbol{1}^\top\boldsymbol{w}_i=1,\quad \boldsymbol{w}_i\ge\boldsymbol{0},
\]
双対可行性
\[
\boldsymbol{\lambda}_i\ge\boldsymbol{0},
\]
および相補性条件
\[
\boldsymbol{\lambda}_i\odot\boldsymbol{w}_i=\boldsymbol{0}
\]
が同時に成り立つとき, $\boldsymbol{w}_i$ は下位問題の最適解である.
\par
DFL-OPT-K は,上記の KKT 条件を制約として直接組み込んだ定式化である.一方,DFL-OPT-D は,主問題の可行性と双対可行性に加え,
停留条件 \eqref{eq:dfl_dual_stationarity} と最適値一致条件 \eqref{eq:dfl_dual_value} を課すことで下位問題の最適性を表現している.
強双対性が成立する凸最適化問題においては,これらの条件は KKT 条件と同値であるため,DFL-OPT-D により許容される $(\boldsymbol{w}_i,\mu_i,\boldsymbol{\lambda}_i)$ の集合は,
DFL-OPT-K により許容される集合と一致する.
\par
以上より,DFL-OPT-D(\eqref{eq:dfl_dual_obj}--\eqref{eq:dfl_dual_stationarity})と 
DFL-OPT-K(\eqref{eq:dfl_kkt_obj}--\eqref{eq:dfl_kkt_complementarity})は同一の解集合を持ち,理論的に等価であることが示された.
\end{proof}

\section*{B. DFL-OPT-D における強双対性制約の導出}
\addcontentsline{toc}{section}{B. DFL-OPT-D における強双対性制約の導出}
\par
ここでは,本文 \eqref{eq:dual_value} および \eqref{eq:dual_strong_simplified}(したがって \eqref{eq:dfl_dual_value})の導出を,双対関数 $g_i$ の導出を含めて示す.
\par
まず,ラグランジュ関数 \eqref{eq:lagrangian} に基づき,双対関数を
\[
g_i(\mu_i,\boldsymbol{\lambda}_i)
:=\inf_{\boldsymbol{w}_i\in\mathbb{R}^d}\mathcal{L}_i(\boldsymbol{w}_i,\mu_i,\boldsymbol{\lambda}_i)
\]
と定義する.ここで $\mathcal{L}_i$ を $\boldsymbol{w}_i$ で最小化する停留条件は本文 \eqref{eq:dual_stationarity} であり,これより
\[
\tilde{\boldsymbol{w}}_i(\mu_i,\boldsymbol{\lambda}_i)
=\frac{1}{\delta}\boldsymbol{V}_i^{-1}\!\left((1-\delta)\hat{\boldsymbol{r}}_i+\mu_i\boldsymbol{1}+\boldsymbol{\lambda}_i\right)
\]
を得る.これを代入すると,双対関数は
\[
g_i(\mu_i,\boldsymbol{\lambda}_i)
=\mu_i-\frac{1}{2\delta}
\left((1-\delta)\hat{\boldsymbol{r}}_i+\mu_i\boldsymbol{1}+\boldsymbol{\lambda}_i\right)^\top
\boldsymbol{V}_i^{-1}
\left((1-\delta)\hat{\boldsymbol{r}}_i+\mu_i\boldsymbol{1}+\boldsymbol{\lambda}_i\right)
\]
と表される.
\par
次に,停留条件 \eqref{eq:dual_stationarity} を
\[
\delta \boldsymbol{V}_i \boldsymbol{w}_i=(1-\delta)\hat{\boldsymbol{r}}_i+\mu_i\boldsymbol{1}+\boldsymbol{\lambda}_i
\]
と書き直し,右辺を $z_i$ とおくと $z_i=\delta \boldsymbol{V}_i\boldsymbol{w}_i$ である.これを上式に代入すると
\[
g_i(\mu_i,\boldsymbol{\lambda}_i)
=\mu_i-\frac{1}{2\delta}(\delta \boldsymbol{V}_i\boldsymbol{w}_i)^\top \boldsymbol{V}_i^{-1}(\delta \boldsymbol{V}_i\boldsymbol{w}_i)
=\mu_i-\frac{\delta}{2}\boldsymbol{w}_i^\top \boldsymbol{V}_i \boldsymbol{w}_i
\]
となり,双対目的値は停留条件を介して $V_i^{-1}$ を含まない形に簡約できる.
\par
弱双対性より,任意の主問題の実行可能解 $\boldsymbol{w}_i$ と双対変数 $(\mu_i,\boldsymbol{\lambda}_i)$($\boldsymbol{\lambda}_i\ge\boldsymbol{0}$)に対して,
主問題目的値は双対目的値の上界である.したがって,
\[
-(1-\delta)\hat{\boldsymbol{r}}_i^\top \boldsymbol{w}_i
+\frac{\delta}{2}\boldsymbol{w}_i^\top \boldsymbol{V}_i \boldsymbol{w}_i
\ge
\mu_i-\frac{\delta}{2}\boldsymbol{w}_i^\top \boldsymbol{V}_i \boldsymbol{w}_i
\]
が成り立つ.これが本文 \eqref{eq:dual_value} である.両辺に $\frac{\delta}{2}\boldsymbol{w}_i^\top \boldsymbol{V}_i \boldsymbol{w}_i$ を加えると
\[
\delta \boldsymbol{w}_i^\top \boldsymbol{V}_i \boldsymbol{w}_i-(1-\delta)\hat{\boldsymbol{r}}_i^\top \boldsymbol{w}_i
\ge \mu_i
\]
を得る.
\par
最後に,強双対性より最適解においては弱双対性の不等式が等号で成立する.このとき KKT 条件の相補性 $\boldsymbol{\lambda}_i\odot\boldsymbol{w}_i=\boldsymbol{0}$ が成り立ち,
導出中に現れる $\boldsymbol{\lambda}_i^\top\boldsymbol{w}_i$ が消えるため,最適値一致条件は
\[
\delta \boldsymbol{w}_i^\top \boldsymbol{V}_i \boldsymbol{w}_i-(1-\delta)\hat{\boldsymbol{r}}_i^\top \boldsymbol{w}_i
= \mu_i
\]
と表される.これが本文 \eqref{eq:dual_strong_simplified} の等式であり,
DFL-OPT-D では各 $i$ についてこの関係を制約 \eqref{eq:dfl_dual_value} として課している.

\section*{C. OAS 縮小係数と実効サンプルサイズ}
\addcontentsline{toc}{section}{C. OAS 縮小係数と実効サンプルサイズ}
\par
ここでは,第4章で用いた OAS の縮小係数 $\phi_t$ と,EWMA の重みに対応する実効サンプルサイズ $n_{\mathrm{eff}}$ の具体式を示す.
\par
まず,$d$ 次元の共分散推定に用いるサンプルサイズを $n_{\mathrm{eff}}$ とすると,OAS による縮小係数は
\begin{equation}
\phi_t
= \min\left\{
1,\ 
\frac{\left(1-\frac{2}{d}\right)\mathrm{tr}(\boldsymbol{S}_t^2)+\mathrm{tr}(\boldsymbol{S}_t)^2}
{\left(n_{\mathrm{eff}}+1-\frac{2}{d}\right)\left(\mathrm{tr}(\boldsymbol{S}_t^2)-\frac{\mathrm{tr}(\boldsymbol{S}_t)^2}{d}\right)}
\right\}
\tag{A.1}
\end{equation}
で与えられる.
\par
次に,EWMA における減衰率 $\alpha$ は,重み付き標本に基づく共分散推定の「記憶長」を制御するパラメータとして解釈できる \cite{jpm2006riskmetrics}.
特に,重み $w_k=(1-\alpha)\alpha^k$ に対して定義される実効サンプルサイズ
\[
\mathrm{ESS}:=\frac{1}{\sum_k w_k^2}
\]
を用いると,EWMA はおよそ $\mathrm{ESS}\approx \frac{1+\alpha}{1-\alpha}$ 個の独立標本に
相当する情報量を持つと解釈できる \cite{jpm2006riskmetrics}.
本研究ではこの対応関係に基づき,短期ローリング設定における数値的安定性と反応性のバランスを考慮して $\alpha$ を設定する.
\par
本研究のように有限窓 $L$ を用いるとき,重みを正規化した $\tilde{w}_k := \frac{(1-\alpha)\alpha^k}{1-\alpha^L}$($k=0,\dots,L-1$)に対して
\begin{equation}
n_{\mathrm{eff}}
:= \frac{1}{\sum_{k=0}^{L-1}\tilde{w}_k^2}
= \frac{(1-\alpha^L)^2}{(1-\alpha)^2}\cdot\frac{1-\alpha^2}{1-\alpha^{2L}}
= \frac{1+\alpha}{1-\alpha}\cdot\frac{(1-\alpha^L)^2}{1-\alpha^{2L}}
\tag{A.2}
\end{equation}
となる.特に $L$ が十分大きい場合には $n_{\mathrm{eff}} \approx \frac{1+\alpha}{1-\alpha}$ となり,上の $\mathrm{ESS}$ の近似と整合する.

\section*{D. 初期解+罰則項あり(全期間)の参考結果}
\addcontentsline{toc}{section}{D. 初期解+罰則項あり(全期間)の参考結果}
\par
本付録では,初期解+罰則項を導入した場合について,
取引コストを控除した実行ベースの損益系列に基づく全期間(2006--2025)の累積資産推移を参考として示す.
なお,本結果は提案手法の主張を補強するものではなく,
探索経路制御が実務的なコストを考慮しても致命的な性能劣化を招かないことを確認する目的で提示する.
\par
本実験では,ティッカー別の取引コスト(bps)を
``SPY:5, GLD:10, EEM:10, TLT:5''
として設定した(デフォルト設定).

\begin{figure}[H]
\centering
\includegraphics[width=0.95\linewidth]{figs/appendix_fullperiod_wealth_penalty.png}
\caption{全期間(2006--2025)における累積資産推移(初期解+罰則項, 取引コスト控除後)}
\label{fig:appendix_fullperiod_wealth_penalty}
\end{figure}

\begin{table}[H]
\centering
\caption{全期間(2006--2025)における性能指標(初期解+罰則項, 取引コスト控除後;参考,太字:1位, 下線:2位. 提案手法は網掛け.)}
\label{tab:appendix_fullperiod_summary_cost}
\footnotesize
\setlength{\tabcolsep}{4pt}
\begin{tabular}{lrrrrr}
\hline
Model & 年率リターン(\%) & 最終資産 & Sharpe 比 & 年率ボラ(\%) & CVaR$_{95}$(\%) \\
\hline
\rowcolor{gray!10}
DFL-OPT-K & \textbf{13.95} & \textbf{12.15} & \textbf{0.83} & \underline{16.76} & \underline{$-$5.21} \\
DFL-CF & 9.66 & 5.07 & 0.56 & 17.36 & $-$5.65 \\
IPO-GRAD & 9.85 & 5.25 & 0.57 & 17.39 & $-$5.63 \\
SPO+ & 9.50 & 4.90 & 0.55 & 17.38 & $-$5.67 \\
PFL & 5.61 & 2.11 & 0.29 & 19.04 & $-$6.68 \\
Buy\&Hold(SPY) & \underline{10.35} & \underline{5.64} & 0.57 & 18.14 & $-$6.19 \\
1/N & 7.13 & 3.64 & \underline{0.62} & \textbf{11.42} & \textbf{$-$3.65} \\
\hline
\end{tabular}
\end{table}

%========================
% Appendix E: 命題の導出
%========================
\section*{E. 命題 \ref{prop:local_sensitivity} の導出(アクティブ集合固定下)}
\addcontentsline{toc}{section}{E. 命題 \ref{prop:local_sensitivity} の導出(アクティブ集合固定下)}
\par
本節では,命題 \ref{prop:local_sensitivity} の導出を示す.
以下では簡単のため時点添字を省略し,
\eqref{eq:lower_qp_sensitivity} の下位問題
\[
\min_{\boldsymbol{w}}
\;f(\boldsymbol{w};\hat{\boldsymbol{r}},\boldsymbol{V})
:=
-(1-\delta)\hat{\boldsymbol{r}}^\top \boldsymbol{w}
+\frac{\delta}{2}\boldsymbol{w}^\top \boldsymbol{V}\boldsymbol{w}
\quad
\text{s.t.}\quad
\boldsymbol{1}^\top\boldsymbol{w}=1,\;\boldsymbol{w}\ge \boldsymbol{0}
\]
を考える.ここで $\delta>0$,$\boldsymbol{V}\succ\boldsymbol{0}$ とする.

\subsection*{E.1 アクティブ集合固定による縮約}
\par
最適解を $\boldsymbol{w}^*(\hat{\boldsymbol{r}},\boldsymbol{V})$ とし,
正の成分集合(アクティブ集合)を
\[
A:=\{j\in\{1,\dots,d\}\mid w_j^*>0\},\qquad N:=A^c
\]
とおく.
アクティブ集合が近傍で不変(すなわち $w_A^*>0$ かつ $w_N^*=0$ が近傍で維持される)と仮定する.
このとき,制約 $w\ge0$ は局所的に
\[
\boldsymbol{w}_N=\boldsymbol{0}
\]
という等式制約に置き換えられる.
したがって,局所的には次の縮約問題と同値になる:
\begin{equation}
\min_{\boldsymbol{w}_A\in\mathbb{R}^{|A|}}
\;-(1-\delta)\hat{\boldsymbol{r}}_A^\top \boldsymbol{w}_A
+\frac{\delta}{2}\boldsymbol{w}_A^\top \boldsymbol{V}_{AA}\boldsymbol{w}_A
\quad
\text{s.t.}\quad
\boldsymbol{1}_A^\top \boldsymbol{w}_A=1.
\label{eq:reduced_qp}
\end{equation}
ここで $\boldsymbol{1}_A$ は $|A|$ 次元の全成分 $1$ ベクトルである.
$\boldsymbol{V}\succ0$ より主小行列 $\boldsymbol{V}_{AA}\succ0$ であり,
\eqref{eq:reduced_qp} は強凸な等式制約付きQPである.

\subsection*{E.2 縮約問題のKKT条件}
\par
\eqref{eq:reduced_qp} のラグランジュ関数を
\[
\mathcal{L}(\boldsymbol{w}_A,\mu)
=
-(1-\delta)\hat{\boldsymbol{r}}_A^\top \boldsymbol{w}_A
+\frac{\delta}{2}\boldsymbol{w}_A^\top \boldsymbol{V}_{AA}\boldsymbol{w}_A
+\mu(1-\boldsymbol{1}_A^\top \boldsymbol{w}_A)
\]
とする.KKT 条件は
\begin{align}
\delta \boldsymbol{V}_{AA}\boldsymbol{w}_A - (1-\delta)\hat{\boldsymbol{r}}_A - \mu \boldsymbol{1}_A &= \boldsymbol{0},
\label{eq:kkt_reduced_stationarity}\\
\boldsymbol{1}_A^\top \boldsymbol{w}_A &= 1
\label{eq:kkt_reduced_budget}
\end{align}
である.
\eqref{eq:kkt_reduced_stationarity} より
\begin{equation}
\boldsymbol{w}_A
=
\frac{1}{\delta}\boldsymbol{V}_{AA}^{-1}
\bigl((1-\delta)\hat{\boldsymbol{r}}_A+\mu \boldsymbol{1}_A\bigr).
\label{eq:wA_mu}
\end{equation}
これを \eqref{eq:kkt_reduced_budget} に代入すると
\[
\boldsymbol{1}_A^\top \boldsymbol{w}_A
=
\frac{1}{\delta}\boldsymbol{1}_A^\top\boldsymbol{V}_{AA}^{-1}
\bigl((1-\delta)\hat{\boldsymbol{r}}_A+\mu \boldsymbol{1}_A\bigr)
=1,
\]
すなわち
\begin{equation}
\mu
=
\frac{\delta-(1-\delta)\boldsymbol{1}_A^\top\boldsymbol{V}_{AA}^{-1}\hat{\boldsymbol{r}}_A}
{\boldsymbol{1}_A^\top\boldsymbol{V}_{AA}^{-1}\boldsymbol{1}_A}.
\label{eq:mu_closed}
\end{equation}
\eqref{eq:mu_closed} を \eqref{eq:wA_mu} に代入すると
\[
\boldsymbol{w}_A^*(\hat{\boldsymbol{r}}_A,\boldsymbol{V}_{AA})
=
\underbrace{\frac{1-\delta}{\delta}\boldsymbol{V}_{AA}^{-1}\hat{\boldsymbol{r}}_A}_{\text{線形項}}
+
\underbrace{\frac{1}{\delta}\boldsymbol{V}_{AA}^{-1}\boldsymbol{1}_A\,
\frac{\delta-(1-\delta)\boldsymbol{1}_A^\top\boldsymbol{V}_{AA}^{-1}\hat{\boldsymbol{r}}_A}
{\boldsymbol{1}_A^\top\boldsymbol{V}_{AA}^{-1}\boldsymbol{1}_A}}_{\text{アフィン補正}}
\]
を得る.よって $\boldsymbol{w}_A^*$ は $\hat{\boldsymbol{r}}_A$ に関して affine である.

\subsection*{E.3 ヤコビアンの導出(式 \eqref{eq:jacobian_affine})}
\par
\eqref{eq:wA_mu} を $\hat{\boldsymbol{r}}_A$ で微分すると
\[
\frac{\partial \boldsymbol{w}_A^*}{\partial \hat{\boldsymbol{r}}_A}
=
\frac{1-\delta}{\delta}\boldsymbol{V}_{AA}^{-1}
+
\frac{1}{\delta}\boldsymbol{V}_{AA}^{-1}\boldsymbol{1}_A
\frac{\partial \mu}{\partial \hat{\boldsymbol{r}}_A}.
\]
一方,\eqref{eq:mu_closed} より
\[
\mu
=
\frac{\delta}{\boldsymbol{1}_A^\top\boldsymbol{V}_{AA}^{-1}\boldsymbol{1}_A}
-
\frac{1-\delta}{\boldsymbol{1}_A^\top\boldsymbol{V}_{AA}^{-1}\boldsymbol{1}_A}
\boldsymbol{1}_A^\top\boldsymbol{V}_{AA}^{-1}\hat{\boldsymbol{r}}_A
\]
なので
\[
\frac{\partial \mu}{\partial \hat{\boldsymbol{r}}_A}
=
-\frac{1-\delta}{\boldsymbol{1}_A^\top\boldsymbol{V}_{AA}^{-1}\boldsymbol{1}_A}
\boldsymbol{1}_A^\top\boldsymbol{V}_{AA}^{-1}.
\]
したがって
\begin{align}
\frac{\partial \boldsymbol{w}_A^*}{\partial \hat{\boldsymbol{r}}_A}
&=
\frac{1-\delta}{\delta}\boldsymbol{V}_{AA}^{-1}
-
\frac{1-\delta}{\delta}
\boldsymbol{V}_{AA}^{-1}\boldsymbol{1}_A
(\boldsymbol{1}_A^\top\boldsymbol{V}_{AA}^{-1}\boldsymbol{1}_A)^{-1}
\boldsymbol{1}_A^\top\boldsymbol{V}_{AA}^{-1},
\end{align}
すなわち本文 \eqref{eq:jacobian_affine} を得る.

\subsection*{E.4 局所Lipschitz上界(式 \eqref{eq:lipschitz_bound})}
\par
同一アクティブ集合 $A$ が維持される局所では
\[
\Delta \boldsymbol{w}_A^*
=
\left(\frac{\partial \boldsymbol{w}_A^*}{\partial \hat{\boldsymbol{r}}_A}\right)\Delta\hat{\boldsymbol{r}}_A
\]
が成り立つため,
\[
\|\Delta \boldsymbol{w}_A^*\|_2
\le
\left\|
\frac{\partial \boldsymbol{w}_A^*}{\partial \hat{\boldsymbol{r}}_A}
\right\|_2
\|\Delta \hat{\boldsymbol{r}}_A\|_2
\]
である.
ここで
\[
\boldsymbol{P}_A
:=
\boldsymbol{I}
-
\boldsymbol{V}_{AA}^{-1}\boldsymbol{1}_A
(\boldsymbol{1}_A^\top\boldsymbol{V}_{AA}^{-1}\boldsymbol{1}_A)^{-1}
\boldsymbol{1}_A^\top
\]
とおくと,
\[
\frac{\partial \boldsymbol{w}_A^*}{\partial \hat{\boldsymbol{r}}_A}
=
\frac{1-\delta}{\delta}
\boldsymbol{P}_A\boldsymbol{V}_{AA}^{-1}.
\]
よって
\[
\left\|
\frac{\partial \boldsymbol{w}_A^*}{\partial \hat{\boldsymbol{r}}_A}
\right\|_2
\le
\frac{1-\delta}{\delta}\,
\|\boldsymbol{P}_A\|_2\,
\|\boldsymbol{V}_{AA}^{-1}\|_2
=
\frac{1-\delta}{\delta}\,
\|\boldsymbol{P}_A\|_2\,
\frac{1}{\lambda_{\min}(\boldsymbol{V}_{AA})}.
\]
これより
\[
\|\Delta \boldsymbol{w}_A^*\|_2
\le
C_A\,
\frac{1-\delta}{\delta}\,
\frac{1}{\lambda_{\min}(\boldsymbol{V}_{AA})}\,
\|\Delta \hat{\boldsymbol{r}}_A\|_2
\]
が得られる.ただし $C_A:=\|\boldsymbol{P}_A\|_2<\infty$ とおいた.
最後に $\Delta \boldsymbol{w}^*$ は $N$ 成分がゼロのままなので,
$\|\Delta \boldsymbol{w}^*\|_2=\|\Delta \boldsymbol{w}_A^*\|_2$ が成り立ち,
本文の \eqref{eq:lipschitz_bound} が従う.
\par
以上で命題 \ref{prop:local_sensitivity} の導出を完了する.

\par
\noindent\textbf{注記(局所性)}:
本節の導出は,ある最適解近傍でアクティブ集合 $A$ が不変である局所に限定している.
境界($w_j^*=0$ に接する点)では集合が切り替わり得るため,写像は一般に大域的には滑らかでない.
\end{document}
